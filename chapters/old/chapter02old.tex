\chapter{Measuring Climate Risk Factors}

A key challenge in implementing a dynamic 
hedging strategy for climate risk is to 
construct a time series that captures 
news about long-term climate risk.

We can start from the observation that 
when there are events that 
plausibly contains information about changes 
in climate risk, this will likely leads 
to newspaper coverage of these events.
Newspapers may even be the direct source 
that investors use to update their beliefs
about climate risk.

\section{Representing Text as Data}
 
Text is high dimensional. 
Suppose we have a bunch of documents, each of 
which is $w$ words long. Each word is 
drawn from a vocabulary of $p$ possible 
words. The unique representation of these 
documents has dimension $p^w$. 

Analysis can 
summarized in three steps:

\begin{enumerate}
    \item Represent raw text $D$ as a numerical 
    array $C$
    \item Map $C$ to predicted 
    values $\hat{V}$ of unknown outcomes 
    $V$
    \item Use $\hat{V}$ in subsequent analysis
\end{enumerate}


The first step in constructing $C$ is 
to divide the raw text $D$ into individual documents 
$D_i$. The way to divide the raw text 
is dictated by the value of interest $V$. 
If $V$ is daily stock price, it might makes sense
to divide the raw text into daily news articles.

To begin with the transformation from raw text $D$ to
a numerical array $C$, we can first count 
the number of times each word appears in each document, $c_{i,j}$.
It results into a matrix $C$ of size $n \times p$ where $n$ is the number of documents and $p$ is 
the number of unique words in the vocabulary.
Each row of $C$ refers to a document $i$, and each column refers to a word $j$.

\begin{examplebox}
    \textbf{Example X.1.} 
\begin{table}[H]
    \centering
    \begin{tabular}{|c|p{5cm}|}
        $D_1$ & a rose is still a rose \\
        $D_2$ & there is no there there \\
        $D_3$ & rose is a rose is a rose is a rose \\
    \end{tabular}
    \caption{Examples of individual documents $D_i$}
\end{table}

\begin{table}[H]
    \centering
    \begin{tabular}{c|c|c|c|c|c|c}
        $i\setminus j$ & a & rose & is & still & there & no \\
        \hline
        $1$ & 2 & 2 & 1 & 1 & 0 & 0 \\
        $2$ & 0 & 0 & 1 & 0 & 3 & 1  \\
        $3$ & 3 & 4 & 3 & 0 & 0 & 0  \\
    \end{tabular}
    \caption{Term frequency matrix $C$}
\end{table}
\end{examplebox}



\section{Dictionary-based Mapping}

Dictionary-based methods are used to map the
counts $c_i$ to outcomes $v_i$.
It specify $\hat{v}_i = f(c_i)$ where $f$ is a 
function pre-specified.
Dictionary-based methods heavily rely 
on prior information about the function mapping 
$c_i$ to outcomes $v_i$.
They are more appropriate when prior information 
is strong and reliable and where information in 
the data is weak.
An example is a case where the outcomes $v_i$ 
are not observed for any $i$, so there is 
no training data available. 

\begin{examplebox}
    \textbf{Example X.1.} 
    Suppose we have a dictionary-based method 
    that maps the counts $c_i$ of
    to outcomes $v_i$.
    The dictionary is a list of words for 
    a specific category.

    \begin{table}[H]
        \centering
        \begin{tabular}{|c|c|}
            Category & Dictionary \\
            \hline
            Positive & good, great, excellent \\
            Negative & bad, terrible, awful \\
        \end{tabular}
        \caption{Example of a dictionary-based method}
    \end{table}

    We have the following documents $D_i$:

    \begin{table}[H]
        \centering
        \begin{tabular}{|c|c|}
            $D_1$ & good is great \\
            $D_2$ & bad is terrible \\
            $D_3$ & good is bad \\
        \end{tabular}
        \caption{Example of documents $D_i$}
    \end{table}

    The matrix $C$ is:

    \begin{table}[H]
        \centering
        \begin{tabular}{c|c|c|c|c|c|c}
            $i\setminus j$ & good & great & bad & terrible & is & \\
            \hline
            $1$ & 1 & 1 & 0 & 0 & 1  \\
            $2$ & 0 & 0 & 1 & 1 & 1  \\
            $3$ & 1 & 0 & 1 & 0 & 1  \\
        \end{tabular}
        \caption{Term frequency matrix $C$}
    \end{table}

    Mapped to the dictionary, it becomes:

    \begin{table}[H]
        \centering
        \begin{tabular}{c|c|c}
            $i\setminus k$ & Positive & Negative \\
            \hline
            $1$ & 2 & 0  \\
            $2$ & 0 & 2  \\
            $3$ & 1 & 1  \\
        \end{tabular}
        \caption{Mapped matrix $C$}
    \end{table}

    We define the function $f(c_i)$ as:

    \begin{equation}
        f(c_i) = \begin{cases}
            \text{Positive} & \text{if } \sum_j c_{i,j} \in \text{Positive} \\
            \text{Negative} & \text{if } \sum_j c_{i,j} \in \text{Negative} \\
            \text{Neutral} & \text{otherwise}
        \end{cases}
    \end{equation}


\end{examplebox}

