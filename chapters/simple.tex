\chapter{Simple Economic Tracking Portfolio}

The portfolio constructed 
here has unexpected returns with maximum correlation 
with news about future macroeconomic variables.

Empirical finance has a long tradition of explaining 
current returns with other current returns. 

\section{Basics}

\subsection{Modelling Returns}

\subsection{Statistics}

\section{Forming Tracking Portfolio with Regression}


\subsection{Regression}

\subsection{Tracking Portfolio}

A tracking portfolio for any variable $y$ can be obtained 
as the fitted value of a regression of $y$ on a set of 
base asset returns. The portfolio weights for the economic 
tracking portfolio for $y$ are identical to the coefficients of 
an OLS regression. If $y$ happens to be a state variable 
for asset pricing, then a multi-factor model holds with 
one of the factors being the $y$'s tracking portfolio.

However, even if $y$ is not a state variable for asset pricing, 
its tracking portfolio is still an interesting object, since 
it reveals changes in market expectation about $y$.

The following three statements are equivalent description 
of an economic tracking portfolio: 
\begin{enumerate}
    \item the portfolio has the minimum variance out 
    of all portfolios with a given beta (univariate regression coefficient)
    in a regression portfolio return on $y$.
    \item has returns with the maximum possible correlation 
    with $y$.
    \item has the highest $R$-squared in a regression of
    $y$ on the portfolio returns.
\end{enumerate}

These properties come directly from the definition of an 
OLS regression. 

\section{Python Project: Tracking the Market Portfolio}

\section{Conclusion}