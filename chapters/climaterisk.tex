\chapter{Climate Risk}


\section{Expected Utility and Optimal Portfolio}

As with ESG preferences only, 
we start by setting up the utility function of an investor
who cares about climate risk, and then derive the optimal portfolio

\subsection{Investor's Expected Utility}

Let $\tilde{C}_1$ denote climate at time 1, which is unknown at
time 0. The investor utility function is now:

\begin{equation}
    V(\tilde{W}_{1,i}, X_i, \tilde{C}_1) = -\exp{(-A_i \tilde{W}_{1,i} - b_i^T X_i - c_i \tilde{C}_1)}
\end{equation}

where $c_i$ is the investor's climate risk sensitivity.

Taking the expectation of the utility function from period 0,
we get: 

\begin{equation}
    E_0(V(\tilde{W}_{1,i}, X_i, \tilde{C}_1)) = E_0(-\exp{(-A_i W_{0,i} - b_i^T X_i - c_i \tilde{C}_1)})
\end{equation}

Again, we can replace $\tilde{W}_{1,i}$
with the relation $\tilde{W}_{1,i} = W_{0,i}(1 + r_f + X_i^T \tilde{r}_1)$
and define $a_i := A_i W_{0,i}$. 
We still want to make out from the expectation the 
terms that we know about in period 0, and 
reexpress the terms with the expectation as a function 
of the portfolio weights $X_i$. 

\begin{equation}
    \begin{aligned}
        E_0(V(\tilde{W}_{1,i}, X_i, \tilde{C}_1)) = E_0(-\exp{(-A_i W_{0,i} - b_i^T X_i - c_i \tilde{C}_1)}) \\
       = E_0(-\exp{(-a_i(1 + r_f + X_i^T \tilde{r}_1) - b_i^T X_i - c_i \tilde{C}_1)}) \\
       = -\exp{(-a_i(1 + r_f))} E_0(-\exp{(-a_i X_i^T \tilde{r}_1 - b_i^T X_i - c_i \tilde{C}_1)}) \\
       = -\exp{(-a_i(1 + r_f))} E_0(-\exp{(-a_i X_i^T(\tilde{r}_1 + \frac{b_i}{a_i}) - c_i \tilde{C}_1)}) \\
       = -\exp{(-a_i(1 + r_f))} -\exp(a_i X_i^T(E_0(\tilde{r}_1)+\frac{b_i}{a_i}) +\\
       \frac{1}{2}a^2_i X_i^T \text{Var}(\tilde{\epsilon}_1)X_i + a_i c_i X_i^T \text{Cov}(\tilde{\epsilon}_1, \tilde{C}_1) + \frac{1}{2}c_i^2 \text{Var}(\tilde{C}_1)) \\
       = -\exp(-a_i(1 + r_f)) - \exp(-a_i X_i^T(\mu + \frac{b_i}{a_i}) + \\
       \frac{1}{2}a_i^2 X_i^T \Sigma X_i + a_i c_i X_i^T \sigma_{\tilde{\epsilon}_1, \tilde{C}_1} + \frac{1}{2}c_i^2 \sigma^2_{\tilde{C}_1})
    \end{aligned}
\end{equation}

where $\sigma_{\tilde{\epsilon}_1, \tilde{C}_1} = \text{Cov}(\tilde{\epsilon}_1, \tilde{C}_1)$.

\subsection{Optimal Portfolio}

Again, the investor $i$ seeks to maximize its expected utility,
by choosing the optimal portfolio weights $X_i$ at time 0.
We need to find the first order conditions for the optimization problem.

We are going to follow the same steps as in the previous chapter. 


\begin{enumerate}
    \item We combine the exponential terms:
    \begin{equation}
        \begin{aligned}
        E_0(V(\tilde{W}_1, X_i, \tilde{C}_1)) = -\exp(-a_i(1 + r_f) -a_i X_i^T(\mu + \frac{b_i}{a_i}) + \\
        \frac{1}{2}a_i^2 X_i^T \Sigma X_i + a_i c_i X_i^T \sigma_{\tilde{\epsilon}_1, \tilde{C}_1} + \frac{1}{2}c_i^2 \sigma^2_{\tilde{C}_1})
        \end{aligned}
    \end{equation}
    and let $f(X_i)$ denotes the exponent: 
    \begin{equation}
        E_0(V(\tilde{W}_1, X_i, \tilde{C}_1)) = -\exp(f(X_i))
    \end{equation}
    \item To differentiate $f(X_i)$ with respect to $X_i$, 
    we use the chain rule $\frac{\partial h}{\partial X_i} = \frac{\partial h}{\partial f} \frac{\partial f}{\partial X_i}$.
    If $h = - \exp(f)$, then $\frac{\partial h}{\partial f} = -\exp(f)$. Thus: 
    \begin{equation}
        \frac{\partial h}{\partial X_i} = -\exp(f) \frac{\partial f}{\partial X_i}
    \end{equation}
    \item We can again use the rules that $\frac{\partial x^T b}{\partial x} = b$ and 
    $\frac{\partial x^T A x}{\partial x} = 2Ax$:
    \begin{equation}
        \begin{aligned}
            \frac{\partial f}{\partial X_i} = -a_i(\mu + \frac{b_i}{a_i}) + a_i^2 \Sigma X_i + a_i c_i \sigma_{\tilde{\epsilon}_1, \tilde{C}_1} \\
        \end{aligned}
    \end{equation}
    Combining:
    \begin{equation}
        \frac{\partial h}{\partial X_i} = -\exp(f) (-a_i(\mu + \frac{b_i}{a_i}) + a_i^2 \Sigma X_i + a_i c_i \sigma_{\tilde{\epsilon}_1, \tilde{C}_1})
    \end{equation}
    \item We set the derivative to zero:
    \begin{equation}
        \begin{aligned}
            -\exp(f)(-a_i(\mu + \frac{b_i}{a_i}) + a_i^2 \Sigma X_i + a_i c_i \sigma_{\tilde{\epsilon}_1, \tilde{C}_1}) = 0 \\
            -a_i(\mu + \frac{b_i}{a_i}) + a_i^2 \Sigma X_i + a_i c_i \sigma_{\tilde{\epsilon}_1, \tilde{C}_1} = 0 \\
        \end{aligned}
    \end{equation}
    because the exponential term is always positive.
    \item We solve for $X_i$:
    \begin{equation}
        \begin{aligned}
            a_i^2 \Sigma X_i = a_i(\mu + \frac{b_i}{a_i} - c_i \sigma_{\tilde{\epsilon}_1, \tilde{C}_1}) \\
            a_i \Sigma X_i = \mu + \frac{b_i}{a_i} - c_i \sigma_{\tilde{\epsilon}_1, \tilde{C}_1} \\
            \Sigma X_i = \frac{1}{a_i}( \mu + \frac{b_i}{a_i} - c_i \sigma_{\tilde{\epsilon}_1, \tilde{C}_1}) \\
            X_i = \frac{1}{a_i}\Sigma^{-1} ( \mu + \frac{b_i}{a_i} - c_i \sigma_{\tilde{\epsilon}_1, \tilde{C}_1})
        \end{aligned}
    \end{equation}
\end{enumerate}


Again, we will assume that $a_i = a$ for all investors:

\begin{equation}
    X_i = \frac{1}{a}\Sigma^{-1} ( \mu + \frac{b_i}{a} - c_i \sigma_{\tilde{\epsilon}_1, \tilde{C}_1})
\end{equation}

\section{Heterogeneous Climate Risk Expectations, Market 
Portfolio and Expected Returns}


\subsection{Heterogeneous Climate Risk Expectations
and Market Portfolio}

We follow the same process than in the previous 
chapter, now including differences in expectations
$c_i$ 
about climate risk $\tilde{C}_1$.

The $n$th elements of investor $i$'s portfolio weight 
vector $X_i$ is still:

\begin{equation}
    X_{i,n} =  \frac{W_{0,i,n}}{W_{0,i}}
\end{equation}

The total wealth invested in stock $n$ at time 0:


\begin{equation}
    W_{0,i,n} := \int_i W_{0,i,n} di 
\end{equation}

The $n$th element of the market portfolio weight vector $w_m$ is:

\begin{equation}
    w_{m,n} = \frac{W_{0,m,n}}{W_{0,m}}
\end{equation}

We reexpress $W_{0,n}$ in terms of individual investors' wealth 
by using the definition of $W_{0,n}$:

\begin{equation}
    w_{m,n} = \frac{1}{W_0} \int_i W_{0,i,n}di 
\end{equation}

with $W_{0,i,n} = W_{0,i}X_{i,n}$, we can rewrite 
the equation: 

\begin{equation}
    \begin{aligned}
    w_{m,n} = \frac{1}{W_0} \int_i W_{0,i}X_{i,n}di \\
    = \int_i \frac{W_{0,i}}{W_0}X_{i,n}di \\
    = \int_i \omega_i X_{i,n}di \\
    \end{aligned}
\end{equation}

We now plug the optimal portfolio weights $X_i$ we have found 
in the previous section into the equation above
to obtain the market weights $w_m$:

\begin{equation}
    \begin{aligned}
        w_{m,n} = \int_i \omega_i \frac{1}{a} \Sigma^{-1} ( \mu + \frac{b_i}{a} - c_i \sigma_{\tilde{\epsilon}_1, \tilde{C}_1}) di \\
        = \frac{1}{a} \Sigma^{-1} \mu (\int_i \omega_i di) + \frac{1}{a^2} \Sigma^{-1}g (\int_i \omega_i d_i di) - \frac{1}{a} \Sigma^{-1} \sigma_{\tilde{\epsilon}_1, \tilde{C}_1} (\int_i \omega_i c_i di) \\
    \end{aligned}
\end{equation}

We have $\int_i \omega_i di = 1$ and $\int_i \omega_i c_i di := \bar{c} \geq 0$,
the wealth-weighted average expectation about climate risk 
across investors. The market portfolio weights are:

\begin{equation}
    \begin{aligned}
    w_m = \frac{1}{a} \Sigma^{-1} (\mu + \frac{g}{a} \bar{d} - \bar{c} \sigma_{\tilde{\epsilon}_1, \tilde{C}_1}) \\
    = \frac{1}{a} \Sigma^{-1} \mu + \frac{g}{a^2} \Sigma^{-1} \bar{d} - \frac{1}{a} \Sigma^{-1} \bar{c} \sigma_{\tilde{\epsilon}_1, \tilde{C}_1}
    \end{aligned}
\end{equation}

\subsection{Market Portfolio Expected Returns}

Starting from the vector of marekt weights $w_m$, we now 
can solver for $\mu$ the vector of expected returns:


\begin{equation}
    \begin{aligned}
        w_m = \frac{1}{a} \Sigma^{-1} \mu + \frac{g}{a^2} \Sigma^{-1} \bar{d} - \frac{1}{a} \Sigma^{-1} \bar{c} \sigma_{\tilde{\epsilon}_1, \tilde{C}_1} \\
        aw_m = \Sigma^{-1} \mu + \frac{g}{a} \Sigma^{-1} \bar{d} - \Sigma^{-1} \bar{c} \sigma_{\tilde{\epsilon}_1, \tilde{C}_1} \\
        aw_m - \frac{g}{a} \Sigma^{-1} \bar{d} + \Sigma^{-1} \bar{c} \sigma_{\tilde{\epsilon}_1, \tilde{C}_1} = \Sigma^{-1} \mu \\
        \Sigma(aw_m - \frac{g}{a} \bar{d} + \bar{c} \sigma_{\tilde{\epsilon}_1, \tilde{C}_1}) = \mu \\
        \mu = a\Sigma w_m - \frac{g}{a} \Sigma \Sigma^{-1} \bar{d} + \bar{c} \Sigma \sigma_{\tilde{\epsilon}_1, \tilde{C}_1} \\
        \mu = a\Sigma w_m - \frac{g}{a} \bar{d} + \bar{c} \Sigma \sigma_{\tilde{\epsilon}_1, \tilde{C}_1}
    \end{aligned}
\end{equation}

Multiplying by $w_m$, we find the market equity premium ($\mu_m = w_m^T \mu$):

\begin{equation}
    \begin{aligned}
    \mu_m = a w_m^T \Sigma w_m - \frac{g}{a} w_m^T \bar{d} + \bar{c} w_m^T \Sigma \sigma_{\tilde{\epsilon}_1, \tilde{C}_1} \\
    = a w_m^T \Sigma w_m + \bar{c} w_m^T \Sigma \sigma_{\tilde{\epsilon}_1, \tilde{C}_1} \\
    = a \sigma^2_m + \bar{c} \sigma_{mC}
    \end{aligned}
\end{equation}

where we still maintain the assumption of an ESG-neutral market portfolio ($w_m^Tg = 0$),
and we have the market portfolio variance 
$\sigma^2_m = w_m^T \Sigma w_m$ and 
$w_m^T \sigma_{\tilde{\epsilon}_1, \tilde{C}_1} = \text{Cov}(\tilde{\epsilon}_1, \tilde{C}_1) = \sigma_{mC}$.

\subsection{Expected Returns with Climate Risk}


We use the last equation in the previous section 
to solve for $a$:

\begin{equation}
    \begin{aligned}
        \mu_m = a \sigma^2_m + \bar{c} \sigma_{mC} \\
        a = \frac{\mu_m - \bar{c} \sigma_{mC}}{\sigma^2_m}
    \end{aligned}
\end{equation}

Then, the expected excess returns can be reexpressed as:

\begin{equation}
    \begin{aligned}
        \mu = a\Sigma w_m - \frac{g}{a} \bar{d} + \bar{c} \sigma_{mC} \\
        = \frac{\mu_m - \bar{c} \sigma_{mC}}{\sigma^2_m} \Sigma w_m - \frac{g}{a} \bar{d} + \bar{c} \sigma_{mC} \\
    \end{aligned}
\end{equation}

We know that $\frac{1}{\sigma^2_m}\Sigma w_m = \beta_m$, 
the market beta:

\begin{equation}
    \begin{aligned}
        \mu =  \frac{\mu_m - \bar{c} \sigma_{mC}}{\sigma^2_m} \Sigma w_m - \frac{g}{a} \bar{d} + \bar{c} \sigma_{mC} \\
        = (\mu_m - \bar{c} \sigma_{mC})\beta_m - \frac{g}{a} \bar{d} + \bar{c} \sigma_{mC}\\
        = \mu_m \beta_m - \bar{c} \sigma_{mC} \beta_m - \frac{g}{a} \bar{d} + \bar{c} \sigma_{mC} \\
        = \mu_m \beta_m - \frac{g}{a} \bar{d} + \bar{c}(\sigma_{mC} - \beta_m \sigma_{mC})
    \end{aligned}
\end{equation}

We know that $\beta_m = (\frac{1}{\sigma^2_m}\sigma_{\tilde{\epsilon}_1, m})$:

\begin{equation}
    \begin{aligned}
        \mu = \mu_m \beta_m - \frac{g}{a} \bar{d} + \bar{c}(\sigma_{mC} - \beta_m \sigma_{mC}) \\
        = \mu_m \beta_m - \frac{g}{a} \bar{d} + \bar{c}(\sigma_{mC} -  \frac{1}{\sigma^2_m}\sigma_{\tilde{\epsilon}_1, m} \sigma_{m, C}) \\
    \end{aligned}
\end{equation}


\begin{equation}
    \mu = \mu_m \beta_m - \frac{\bar{d}}{a}g + \bar{c}(1 - \rho^2_{mC}) \psi 
\end{equation}

Expected returns depend on climate betas, $\psi$, which represent 
firms' exposures to non-market climate risk. A firm's climate 
beta is its loading on $\tilde{C}_1$ after controlling for the market return. 


\section{Utility in Equilibrium with Heterogenous 
Climate Risk Expectations}

\section{Climate Risk Hedging Portfolio}
