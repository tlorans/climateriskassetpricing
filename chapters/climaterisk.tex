\chapter{Climate Risk}


\section{Expected Utility and Optimal Portfolio}

As with ESG preferences only, 
we start by setting up the utility function of an investor
who cares about climate risk, and then derive the optimal portfolio

\subsection{Investor's Expected Utility}

Let $\tilde{C}_1$ denote climate at time 1, which is unknown at
time 0. The investor utility function is now:

\begin{equation}
    V(\tilde{W}_{1,i}, X_i, \tilde{C}_1) = -\exp{(-A_i \tilde{W}_{1,i} - b_i^T X_i - c_i \tilde{C}_1)}
\end{equation}

where $c_i$ is the investor's climate risk sensitivity.

Taking the expectation of the utility function from period 0,
we get: 

\begin{equation}
    E_0(V(\tilde{W}_{1,i}, X_i, \tilde{C}_1)) = E_0(-\exp{(-A_i W_{0,i} - b_i^T X_i - c_i \tilde{C}_1)})
\end{equation}

Again, we can replace $\tilde{W}_{1,i}$
with the relation $\tilde{W}_{1,i} = W_{0,i}(1 + r_f + X_i^T \tilde{r}_1)$
and define $a_i := A_i W_{0,i}$. 
We still want to make out from the expectation the 
terms that we know about in period 0, and 
reexpress the terms with the expectation as a function 
of the portfolio weights $X_i$. 

\begin{equation}
    \begin{aligned}
        E_0(V(\tilde{W}_{1,i}, X_i, \tilde{C}_1)) = E_0(-\exp{(-A_i W_{0,i} - b_i^T X_i - c_i \tilde{C}_1)}) \\
       = E_0(-\exp{(-a_i(1 + r_f + X_i^T \tilde{r}_1) - b_i^T X_i - c_i \tilde{C}_1)}) \\
       = -\exp{(-a_i(1 + r_f))} E_0(-\exp{(-a_i X_i^T \tilde{r}_1 - b_i^T X_i - c_i \tilde{C}_1)}) \\
       = -\exp{(-a_i(1 + r_f))} E_0(-\exp{(-a_i X_i^T(\tilde{r}_1 + \frac{b_i}{a_i}) - c_i \tilde{C}_1)}) \\
    \end{aligned}
\end{equation}

\subsection{Optimal Portfolio}

\section{Expected Returns}

\begin{equation}
    \mu = \mu_m \beta_m - \frac{\bar{d}}{a}g + \bar{c}(1 - \rho^2_{mC}) \psi 
\end{equation}

Expected returns depend on climate betas, $\psi$, which represent 
firms' exposures to non-market climate risk. A firm's climate 
beta is its loading on $\tilde{C}_1$ after controlling for the market return. 

