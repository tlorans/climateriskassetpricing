\chapter{Efficient Porfolio}
 
We now want to improve upon this first portfolio 
by including more sophisticated optimized mimicking 
portfolio construction, by \textbf{taking into account 
the information on the covariance between the 
assets}. Indeed, even though the mimicking portfolio we have 
seen in the previous chapter is very intuitive, 
it can be \textbf{suboptimal depending on the 
risk appetite of the investor}.

\section{Portfolio Optimization}

Let's consider an universe of $N$ assets. 
We have $w = \begin{bmatrix}
    w_1 & w_2 & \ldots & w_N
\end{bmatrix}^T$ the vector of weights in the
portfolio.
We suppose the portfolio is fully invested, i.e.
$w^T \mathbf{1} = 1$.
We have $R = \begin{bmatrix}
    R_1 & R_2 & \ldots & R_N
\end{bmatrix}^T$ the vector of returns of the assets.

The return of the portfolio is given by:

\begin{equation}
    R_p = \sum^N_{i=1} w_i R_i = w^T R
\end{equation}

The vector of expected asset returns is denoted by $\mu = E(R)$.
The covariance matrix of the asset returns is denoted by $\Sigma$:

\begin{equation}
    \Sigma = E[(R - \mu)(R - \mu)^T]
\end{equation}

The variance of the portfolio is given by:

\begin{equation}
    \begin{aligned}
        \sigma^2_p &= E[(R_p - E(R_p))^2] \\
        &= E[(w^T R - w^T \mu)^2] \\
        &= E[(w^T (R - \mu))^2] \\
        &= E[w^T (R - \mu)(R - \mu)^T w] \\
        &= w^T E[(R - \mu)(R - \mu)^T] w \\
        &= w^T \Sigma w
    \end{aligned}
\end{equation}

The problem of the investor can be formulated as:

\begin{enumerate}
    \item \textbf{Maximize the expected return of the portfolio
    under a volatility constraint ($\sigma_p \leq \sigma^*$)}
    \item \textbf{Minimize the volatility of the portfolio under a
    return constraint ($\mu_p \geq R^*$)}
\end{enumerate}

\textbf{The key idea of Markowitz (1956) is to combine 
the two objectives into a single quadratic optimization problem}:

\begin{equation}
    \begin{aligned}
        & \underset{w}{\min}
        & &  \frac{1}{2} w^T \Sigma w - \lambda w^T \mu \\
        & \text{subject to}
        & & w^T \mathbf{1} = 1
    \end{aligned}
\end{equation}
    
where $\lambda$ is a parameter that allows to trade-off
between the two objectives (this is the risk appetite in this case).

If we set $\lambda = 0$, we are minimizing the volatility of the portfolio
and obtain the minimum variance portfolio.
If we set $\lambda = \infty$, we are maximizing the return of the portfolio
without taking into account the volatility of the portfolio.

The exact value of $\lambda$ depends on the risk 
preference of the investor: the higher $\lambda$, the more
higher the risk appetite.

\section{Climate Efficient Mimicking Portfolio}

We can use the derived signal $b$ to compute 
a constrained long-only portfolio ($w_i \geq 0$)
and fully invested ($w^T \mathbf{1} = 1$) mimicking 
portfolio.

The optimization problem is now:
\begin{equation}
    \begin{aligned}
        & \underset{w}{\min}
        & &  \frac{1}{2} w^T \Sigma w - \lambda w^T b \\
        & \text{subject to}
        & & w^T \mathbf{1} = 1 \\
        & & & w_i \geq 0
    \end{aligned}
\end{equation}

We are now using the covariance matrix $\Sigma$ of the assets
to construct the \textit{efficient mimicking portfolio}, 
along the signal $b$.


