\chapter{Efficient Porfolio}

Even though the mimicking portfolio we have 
seen in the previous chapter is very intuitive, 
it is suboptimal as it is inefficient. 
We now want to improve upon this first portfolio 
by including more sophisticated optimized mimicking 
portfolio construction, by taking into account 
the information on the covariance between the 
assets.

\section{Portfolio Optimization}

Let's consider an universe of $N$ assets. 
We have $w = \begin{bmatrix}
    w_1 & w_2 & \ldots & w_N
\end{bmatrix}^T$ the vector of weights in the
portfolio.
We suppose the portfolio is fully invested, i.e.
$w^T \mathbf{1} = 1$.
We have $R = \begin{bmatrix}
    R_1 & R_2 & \ldots & R_N
\end{bmatrix}^T$ the vector of returns of the assets.

The return of the portfolio is given by:

\begin{equation}
    R_p = \sum^N_{i=1} w_i R_i = w^T R
\end{equation}
The vector of expected asset returns is denoted by $\mu = E(R)$.
The covariance matrix of the asset returns is denoted by $\Sigma$:

\begin{equation}
    \Sigma = E[(R - \mu)(R - \mu)^T]
\end{equation}

The variance of the portfolio is given by:

\begin{equation}
    \begin{aligned}
        \sigma^2_p &= E[(R_p - E(R_p))^2] \\
        &= E[(w^T R - w^T \mu)^2] \\
        &= E[(w^T (R - \mu))^2] \\
        &= E[w^T (R - \mu)(R - \mu)^T w] \\
        &= w^T E[(R - \mu)(R - \mu)^T] w \\
        &= w^T \Sigma w
    \end{aligned}
\end{equation}

The problem of the investor can be formulated as:

\begin{enumerate}
    \item Maximize the expected return of the portfolio
    under a volatility constraint ($\sigma_p \leq \sigma^*$)
    \item Minimize the volatility of the portfolio under a
    return constraint ($\mu_p \geq R^*$)
\end{enumerate}

The key idea of Markowitz (1956) is to combine 
the two objectives into a single optimization problem:

\begin{equation}
    \begin{aligned}
        & \underset{w}{\min}
        & &  \frac{1}{2} w^T \Sigma w - \lambda w^T \mu \\
        & \text{subject to}
        & & w^T \mathbf{1} = 1
    \end{aligned}
\end{equation}
    
where $\lambda$ is a parameter that allows to trade-off
between the two objectives (this is the risk appetite in this case).

\section{Efficient Mimicking Portfolio}

We can use the derived signal $b$ to compute 
a constrained long-only portfolio ($w_i \geq 0$)
and fully invested ($w^T \mathbf{1} = 1$) mimicking 
portfolio.

The optimization problem is:
\begin{equation}
    \begin{aligned}
        & \underset{w}{\min}
        & &  \frac{1}{2} w^T \Sigma w - \lambda w^T b \\
        & \text{subject to}
        & & w^T \mathbf{1} = 1 \\
        & & & w_i \geq 0
    \end{aligned}
\end{equation}



\section{Performance Measures}
