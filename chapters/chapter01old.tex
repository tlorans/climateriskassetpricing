\chapter{Uncovering Climate Risk with Climate News}

Because of the long-term nature of climate risk,
standard futures or insurance 
contracts in which one party agrees 
to pay the other in the event of a climate 
disaster are difficult to implement. 
Rather than buying a security that pays off
in the event of a climate disaster, 
you can construct 
portfolio whose short-term returns hedge
\textit{news} about climate risk.
By hedging, period by period, the innovations 
in news about long-run climate risk, 
an investor can ultimately hedge her 
long-run exposure to climate risk.

\section{Climate Risk}

Climate risk is a long-run risk.
It is usually separated into two components:
\begin{itemize}
    \item \textbf{Physical risk}: the risk that 
    climate change will have direct effects on 
    the value of assets. For example, 
    a rise in sea levels could affect the value 
    of real estate.
    \item \textbf{Transition risk}: the risk that 
    the transition to a low-carbon economy will 
    affect the value of assets. For example, 
    a carbon tax could affect the value of 
    fossil fuel companies.
\end{itemize}

While global warming already has physical effects,
most of the potential damages are still in the future.
The transition to a low-carbon economy is also
a long-run process, with a lot of uncertainty
about the timing and the nature of the transition.

CLIMATE SCENARIOS.

$CC_{t+\tau}$ is the climate risk at time $t+\tau$.

\section{Uncovering Long-Run Climate Risk
with Changes in Expectation}

We are facing a strong uncertainty about 
$CC_{t+\tau}$, the climate risk at time $t+\tau$. 
Climate risk is not a tradable asset.

We can draw inspiration from the literature
on \textit{factor mimicking}, 
following Lamont (2001) \cite{lamont2001economic}.
We first explain how can we mimic the behavior 
of a non-tradable factor signal $y$ (think about industrial 
production, inflation, \textit{etc.}) with a portfolio of tradable assets.


\subsection{Portfolio Mimicking of Non-Tradable Assets}

The first core idea is to replace some variables with 
a linear combination of other variables. More
specifically, \textbf{some variable of interest can be
written as a portfolio of \textit{tradable assets}}.
It can be used to proxy economic 
variable that 
are not directly observable with tradable assets.
We can construct, from financial assets, 
a "matching portfolio" of some economic factor
that is
not directly tradable.


Say you want to estimate current (this is \textit{nowcasting})
GDP or inflation. You can construct the portfolio 
of assets that best mimics the movements of GDP or inflation.
Once you've run your regression, you can use your 
estimate of returns to predict the macro variable.
Let's say we don't have individual 
stock returns and we want to estimate 
the market return. All what we have is
the return of $K$ industry portfolios. We 
can estimate the market return as a 
linear combination of the industry returns:

\begin{equation}
R_{m,t} = \beta_{\text{energy}} R_{\text{energy},t} + \beta_{\text{financials}} R_{\text{financials},t} + \ldots + \beta_{k} R_{k,t} + \epsilon_{t}
\end{equation}

It looks very much like a portfolio, with 
the estimated $\beta$ as the weights of the assets.


\subsection{Unexpected Returns and Changes in Expectations}

The second core idea is that 
we can \textbf{uncover expectations about any variable
correlated with future cash flows and discount rates
by looking at the unexpected returns of tradable assets.}

For climate risk, let's start by defining the current expectation
of $CC_{t+\tau}$ as:

\begin{equation}
    E_{t}(CC_{t+\tau}) = E_{t-1}(CC_{t+\tau}) + \Delta E_{t}(CC_{t+\tau})
\end{equation}

That is, the current expectation is the previous
expectation plus the news or "surprise" about the climate risk.
On a similar vein, we can define realized returns as:

\begin{equation}
    R_{t} = E_{t-1}(R_{t}) + \tilde{R}_{t}
\end{equation}

with $\tilde{R}_{t}$ the unexpected returns.
It's simply the difference between the 
expected returns and the actual returns
$\tilde{R}_{t} = R_{t} - E_{t-1}(R_{t})$.

Our key assumption now is that \textbf{innovations 
in returns (unexpected returns) reflect innovations
in expectations about the climate risk}, such that:

\begin{equation}
    \Delta E_{t}(CC_{t+\tau}) = \beta^T \tilde{R}_{t} + \epsilon_{t}
\end{equation}

If the climate risk $CC_{t+\tau}$ is correlated
with future cash flows and discount rates,
then we may find something in the $\beta$, relating
news reflected in the unexpected returns.
This is based on the assumption that the unexpected
returns reflect news about the future cash flows
and discount rates (\textit{i.e.} about
$\Delta E_{t}(CC_{t+\tau})$).



% Now, let's say we want to form a portfolio that 
% mimics the behavior of a \textit{factor signal} $y$.
% Specifically, our target is "news" or \textit{innovations} about the signal,
% defined as the difference between the current expectation 
% and the previous expectation:

% \begin{equation}
%     \Delta E_t(y_{t+1}) := E_t(y_{t+1}) - E_{t-1}(y_{t+1})
% \end{equation}

% In can be for example the news that the market 
% learns about the industrial production in May about 
% the industrial production in June.

% The mimicking portfolio portfolio returns are:

% \begin{equation}
%     R_{y, t} = w^T R_t
% \end{equation}

% where $R_t$ is the vector of returns of the tradable assets.
% You construct the mimicking portfolio with 
% \textit{unexpected returns} of the tradable assets. Unexpected 
% returns are actual returns minus expected returns:

% \begin{equation}
%     \tilde{R_t} := R_t - E_{t-1}(R_t)
% \end{equation}

% with the assumption that \textbf{the expected returns are a linear 
% function of factors $F_{t}$}:

% \begin{equation}
%     E_{t-1}(R_t) = \gamma^T F_{t}
% \end{equation}


% The porfolio weights of the mimicking portfolio of $y$ 
% are chosen so that $\tilde{R}_{y,t}$ is as close as possible
% to $\Delta E_t(y_{t+1})$ (maximally correlated).

% To do this, the key assumption is that 
% \textbf{innovations in returns (unexpected returns)
% reflect innovations in expectations about 
% the factor signal}, such that:

% \begin{equation}
%     \Delta E_t(y_{t+1}) = \beta^T \tilde{R}_t + \epsilon_t
% \end{equation}

% If the factor signal $y$ is correlated 
% with future cash flows and discount rates,
% then we may find something in the $\beta$, relating
% news reflected in the unexpected returns. 
% Again, this is based on the assumption that 
% the unexpected returns reflect news about the
% future cash flows and discount rates (\textit{i.e.}
% about $\Delta E_t(y_{t+1})$).

% Recalling that the returns are $R_t = E_{t-1}(R_t) + \tilde{R}$,
% we can therefore rewrite it with the factors:

% \begin{equation}
%     R_t = \gamma^T F_{t} + \tilde{R}
% \end{equation}

% and then includes the innovations in the factor signal:

% \begin{equation}
%     R_t = \gamma^T F_{t} + \eta \Delta E_t(y_{t+1}) + u_{t}
% \end{equation}

% What we have here? 
% \textbf{The returns of any asset can be written 
% as a function of its expected returns ($\gamma^T F_{t}$)
% and the unexpected returns. The unexpected returns are 
% decomposed into the news about the factor signal $\Delta E_t (y_{t+1})$ and
% uncorrelated errors ($u_t$).}

\section{Implications about Climate Risk Pricing}

We should be able to estimate the risk premium of climate risk
as the average of the excess returns of the mimicking portfolio. 
However, given the fact that we have a short-time serie, 
the resulting risk premium will be noisy. Indeed, 
investors attention to climate risk is a relatively recent phenomenon.

Unexpected rturns vs. realized returns from Dissecting Green returns:
do no draw conclusions about the expected returns of green assets
from cumulative unexpected returns due to climate change news.

% \section{Climate Factor Signal}

% In the case of climate risk, we have:

% \begin{equation}
%     \Delta E_t (CC_{t+k}) = E_t (CC_{t+k}) - E_{t-1} (CC_{t+k})
% \end{equation}

% with $CC_{t+k}$ the climate risk at an undefined horizon $k$.
% We have seen in the chapter 1 how to use text to proxy for 
% the market expectation of the climate risk $E_t(CC_{t+k})$.


% \begin{examplebox}
%     \textbf{Example X.1.}    
%     \textit{illustrates use of demean vs. AR(1) to estimate}
% \end{examplebox}


% Therefore, we can estimate:

% \begin{equation}
%     R_{i,t} = \beta_i \Delta E_t(CC_{t+k}) + \gamma_i^T \text{Factors}_t + \epsilon_{i,t}
% \end{equation}

% The $\beta_i$ for each portfolio is the signal 
% upon which we want to derive our mimicking portfolio.
% The higher is $\beta_i$, the higher will be the weight 
% of the portfolio in the mimicking portfolio.

% However, if $\beta_i$ is negative, it means that
% we should short the portfolio. In practice, 
% it is less common to short portfolios, so we
% can set $\beta_i = 0$ if it is negative. In that 
% case, the mimicking portfolio would be a \textit{long-only}
% portfolio.

% A simple method to go from the $\beta_i$ to the
% weights of the mimicking portfolio is to normalize it:
% \begin{equation}
%     w_i = \frac{\tilde{\beta_i}}{\sum_{i=1}^{N} \tilde{\beta_i}}
% \end{equation}

% with $\tilde{\beta_i} = \max(\beta_i, 0)$.

% The vector of weights $w$ is already an intuitive 
% long-only mimicking portfolio: it simply 
% \textbf{weights the portfolios based on their positive 
% $\beta_i$ on the climate innovation signal 
% $\Delta E_t(CC_{t+k})$ we wish to mimic}.
 
\section{Conclusion}

Climate risk is a non-tradable asset, with a long-run horizon.
Portfolio mimicking allows for replicating the behavior of
non-tradable assets with tradable assets. Using the unexpected
returns of tradable assets, we can uncover expectations about
non-tradable assets. We can use this framework to hedge
climate risk by constructing a portfolio that hedges
the changes in expectations about climate risk.

But what are changes in expectations about climate risk?
This is the topic of the next chapter.