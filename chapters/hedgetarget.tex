\chapter{Climate Hedge Targets}

One challenge with designing portfolios 
that hedge climate risks is that there is no unique way of choosing the hedge target.
Climate change is a complex phenomenon and presents
a variety of risks, including physical risks such as rising sea levels and transition risks such
as the dangers to certain business models from regulations to curb emissions. 
Different risks may be relevant for different investors, 
and these risks are imperfectly correlated. In
addition, climate change is a long-run threat, 
and we would thus ideally build portfolios that
hedge the long-run realizations of climate risk, 
something difficult to produce in practice.
To overcome these challenges, 
Engle \textit{et al.} (2020) \cite{engle2020hedging} argue that 
the objective of hedging long-run realizations of a given climate risk
can be achieved by constructing a sequence of short-lived hedges against 
\textit{news} (one-period innovation in expectations) about future 
realizations of the risk. Following the initial work of Engle \textit{et al.} (2020),
researchers have developped a variety of climate news series, capturing 
a variety of climate risks. 

\section{Climate News Series}

Describe some climate new series.

\section{Climate News Innovation}

Building on the work of Engle \textit{et al.} (2020), 
we use the $AR(1)$ innovations of each climate news series as the hedge targets.
For a given climate news series $c$, we denote these $AR(1)$ innovation 
in month $t$ as $CC_{c,t}$.

\subsection{Climate News Shock}

\section{Portfolio Exposure to Climate News Innovations}

\subsection{Multifactor Regression}

\subsection{Climate News Innovations as a Risk Factor}