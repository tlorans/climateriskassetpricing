\chapter{Climate Risk}


Hedge target.


\section{Climate Risks and Asset Prices}

A first step is to identify the factors of climate risks.
Why climate risks could have an impact on asset prices?

To get an intuition, we can start from the definition of 
returns in a one period  (\textit{ie.} asset lasts only one period):

\begin{equation}
    R_{t+1} = \frac{D_{t+1}}{P_t}
\end{equation}

with $R_{t+1}$ the return of the asset from time $t$ to time $t+1$, 
$D_{t+1}$ the dividend paid at time $t+1$ 
and $P_t$ the price of the asset at time $t$.
There is no $P_{t+1}$ in the equation, 
as we are in a one period model and 
the stock doesn't exist anymore at time $t+1$. Take the expectations:

\begin{equation}
    E_t(R_{t+1}) = \frac{E_t(D_{t+1})}{P_t}
\end{equation}

And solve for $P_t$:

\begin{equation}
    P_t = \frac{E_t(D_{t+1})}{E_t(R_{t+1})}
\end{equation}

These formula 
represents the price of the asset at time $t$ 
as the discounted value of future dividends.
Dividends are used as a proxy for future cash flows.
The idea is: 

\begin{itemize}
    \item Asset prices are determined by
    expectations of future cash flows or discount rate.
    \item Climate risks may 
    affect expectations of future cash flows or discount rate.
    \item Therefore, climate risks may have an impact on asset prices.
\end{itemize}

Expectations about climate risks should 
already be reflected in today's asset prices.
The expected climate risks should, 
in theory, be already priced in.

Now what drives the changes in asset prices?
Ultimately, we are interested in the changes in asset prices,
source of potential \textit{returns}.
The changes in expectations about 
future cash flows or discount rate. 
So,
we are not so much interested in the level of
expectations of climate risks, but in the changes
in beliefs about climate risks.

Investors form expectations about climate risks
in an horizon $h$ with information available (those 
are conditional expectations).
Each period, new information arrives and investors
update their beliefs about climate risks:

\begin{equation}
    \Delta E_t (CC_{t+h}) = E_t(CC_{t+h}) - E_{t-1}(CC_{t+h})
\end{equation}

with $\Delta E_t (CC_{t+h})$ the \textit{innovation} or 
\textit{news} in climate risk. On the other hand, we have the \textit{unexpected} returns $\tilde{R}_t$:

\begin{equation}
    \tilde{R}_t = R_t - E_{t-1}(R_t)
\end{equation}

with $R_t$ the \textit{realized} returns 
and $E_{t-1}(R_t)$ the expected returns based on information available at time $t-1$.
Because investors reprice assets based on the arrival of new information,
we can expect that the innovation in climate risk is reflected in the unexpected returns:

\begin{equation}
    \tilde{R}_t = \beta \Delta E_t(CC_{t+h}) + \varepsilon_t
\end{equation}


with $\beta$ non-null as long as we expect 
that changes in climate risks affect investors 
expectations about future cash flows or discount rate.


\section{Measuring Climate Risks}

For non-tradable factors such as macro factors, 
creating a time series that capture expectations 
of these factors is not so difficult. You may 
use data from the central bank, the government,
\textit{etc}. They publish leading indicators,
surveys, \textit{etc}. The task is more 
challenging for climate risks.

A common approach in the literature (see Engle et al. (2020) \cite{engle2020hedging})
is to use newspapers coverage of climate events as a proxy 
for the average investor's beliefs about climate risks.
As they noted, when there are events that plausibly
contains information about changes in climate risk,
this will likely leads to newspaper coverage of these events.
Newspapers may even be the direct source that investors
use to update their beliefs about climate risk.


\section{Conclusion}

In what follow we propose a method to construct
hedge portfolios with tradable assets that
mimic the behavior of climate risks. 
