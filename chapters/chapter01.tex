\chapter{Measuring Climate Risks}

Risk is the unexpected in Finance.
Suppose that a stock return follows an AR(1) process:

\begin{equation}
    R_{t} = \phi R_ {t-1} + \varepsilon_{t}
\end{equation}

with $\varepsilon_{t}$ with mean 0 and variance $\sigma_{\varepsilon}^2$.

The conditional expectation of the stock return is:

\begin{equation}
    E_t(R_{t+1}) = \phi R_t
\end{equation}

The actual or \textit{realized} return will vary 
around this expectation. The unexpected return is:

\begin{equation}
    R_{t+1} - E_t(R_{t+1}) = \varepsilon_{t+1}
\end{equation}



\section{Climate Risks and Asset Prices}


A first step is to identify the factors of climate risks.
Why climate risks could have an impact on asset prices?

\begin{itemize}
    \item Asset prices are determined by
    expectations of future cash flows or discount rate.
    \item Climate risks may 
    affect expectations of future cash flows or discount rate.
    \item Therefore, climate risks may have an impact on asset prices.
\end{itemize}

Expectations about climate risks should 
already be reflected in today's asset prices.
The expected climate risks should, 
in theory, be already priced in.

\subsection{Cash-Flows Channel}

Let's define a payoff $X_{t+1}$ as the 
cash-flows provided to the investors per dollar invested in the stock at time $t$:

\begin{equation}
    X_{t+1} = \frac{CF_{t+1}}{P_t}
\end{equation}

We assume a one-period model, with only 
two sources of uncertainty. 
Analysts form forecasts 
about the future cash-flows of the firm
based on information available at time $t$.
The cash-flows of the firm are a function of
two factors: macroeconomic conditions and climate change.
The sensitivity of the firm to these factors are $\beta_m$ and $\beta_{cc}$:

\begin{equation}
    E_t(X_{t+1}) = \beta_m E_t(M_{t+1})
    + \beta_{cc} E_t(CC_{t+1})
\end{equation}

Analysts doesn't perfectly forecast future macroeconomic conditions
and climate change. They form expectations about these factors 
($E_t(M_{t+1})$ and $E_t(CC_{t+1})$).
As every forecasts, they may be wrong, 
and \textit{realized} macroeconomic conditions and climate change
may differ from expectations:

\begin{equation}
    M_{t+1} = E_t(M_{t+1}) + \tilde{M}_{t+1}
\end{equation}

\begin{equation}
    CC_{t+1} = E_t(CC_{t+1}) + \tilde{CC}_{t+1}
\end{equation}

with $\tilde{M}_{t+1}$ and $\tilde{CC}_{t+1}$ the \textit{shocks} to macroeconomic conditions 
and climate change. These shocks are unanticipated
changes.
Due to the possibility of unanticipated shocks,
the realized cash-flows may also differ from expected cash-flows:

\begin{equation}
    X_{t+1} - E_{t}(X_{t+1}) = \beta_m \tilde{M}_{t+1} + \beta_{CC} \tilde{CC}_{t+1} + \varepsilon_{t+1}
\end{equation}

$E_t(X_{t+1})$ is the expected payoff at time $t+1$,
with forecasts based on information available at time $t$.

\subsection{Discount Rate Channel}

Now, how the stock is priced? 
The price of the stock is the discounted value of the 
cash-flows per dollar invested in the stock.
The cash-flows are discounted according 
to a required rate of return for the firm:

\begin{equation}
    P_{t+1} = \frac{X_{t+1}}{\rho_{t+1}}
\end{equation}

If we assume that $\beta_m = 0$,
we have a time-varying discount rate $\rho_{t+1}$:

\begin{equation}
    \rho_{t+1} = 1 - \frac{\beta_{CC}}{\gamma} d_{t+1}
\end{equation}

with $d_{t+1}$ the perception of climate 
risks by the average investor at time $t+1$,
and $\gamma$ the constant risk aversion of 
the average investor.

The price in $P_{t+1}$, once 
the shock are realized at time $t+1$, will be:

\begin{equation}
    P_{t+1} = \frac{X_{t+1}}{1 - \frac{\beta_{CC}}{\gamma}d_{t+1}}
\end{equation}

Which can be approximated as:

\begin{equation}
    P_{t+1} \approx X_{t+1} + \frac{\beta_{CC}}{\gamma}d_{t+1}
\end{equation}

It's expected value at time $t$ is:

\begin{equation}
    E_t(P_{t+1}) = E_t(X_{t+1}) + \frac{\beta_{CC}}{\gamma}E_t(d_{t+1})
\end{equation}

And the unexpected return is:

\begin{equation}
    R_{t+1} - E_t(R_{t+1}) = P_{t+1} - E_t(P_{t+1})
\end{equation}

because $X_{t+1}$ is the payoff per dollar invested in the stock at time $t$,
which corresponds to the definition of a return in a 
one-period model $R_{t+1} = \frac{CF_{t+1}}{P_t}$.

We now can substitute $P_{t+1}$ and $E_t(P_{t+1})$ in the equation above:

\begin{equation}
    R_{t+1} - E_t(R_{t+1}) = X_{t+1} + \frac{\beta_{CC}}{\gamma}d_{t+1} - E_t(X_{t+1}) - \frac{\beta_{CC}}{\gamma}E_t(d_{t+1})
\end{equation}

We recognize the unexpected payoff $X_{t+1} - E_t(X_{t+1})$:

\begin{equation}
    R_{t+1} - E_t(R_{t+1}) = X_{t+1} - E_t(X_{t+1}) + \frac{\beta_{CC}}{\gamma}d_{t+1} - \frac{\beta_{CC}}{\gamma}E_t(d_{t+1})
\end{equation}

We can rewrite the unexpected payoff as a function of the shocks:

\begin{equation}
    R_{t+1} - E_t(R_{t+1}) = \beta_m \tilde{M}_{t+1} + \beta_{CC} \tilde{CC}_{t+1} + \varepsilon_{t+1} + \frac{\beta_{CC}}{\gamma}d_{t+1} - \frac{\beta_{CC}}{\gamma}E_t(d_{t+1})
\end{equation}

Let's drop the macroeconomic component as 
we assumed that $\beta_m = 0$:

\begin{equation}
    R_{t+1} - E_t(R_{t+1}) = \beta_{CC} \tilde{CC}_{t+1} + \varepsilon_{t+1} + \frac{\beta_{CC}}{\gamma}d_{t+1} - \frac{\beta_{CC}}{\gamma}E_t(d_{t+1})
\end{equation}

We can factorize the second part, with
the unexpected change in the average perception of climate risks $d$:

\begin{equation}
    R_{t+1} - E_t(R_{t+1}) = \beta_{CC} \tilde{CC}_{t+1} + \varepsilon_{t+1} + \frac{\beta_{CC}}{\gamma}(d_{t+1} - E_t(d_{t+1}))
\end{equation}

We group together the terms related to the climate change:

\begin{equation}
    R_{t+1} - E_t(R_{t+1}) = \beta_{CC} \tilde{CC}_{t+1} + \varepsilon_{t+1} + \beta_{CC} \left( \frac{1}{\gamma}(d_{t+1} - E_t(d_{t+1})) \right)
\end{equation}

\begin{equation}
    R_{t+1} - E_t(R_{t+1}) = \beta_{CC} \left(\tilde{CC}_{t+1} + \frac{1}{\gamma}(d_{t+1} - E_t(d_{t+1})) \right) + \varepsilon_{t+1}
\end{equation}

What does it says? In our simple model, 
we have related the unexpected returns to the
unexpected changes in climate risks. It can 
either comes from the realization of an 
unexpected climate event $\tilde{CC}_{t+1}$,
or from the unexpected change in the average
perception of climate risks $d_{t+1} - E_t(d_{t+1})$.



Now what drives the changes in asset prices?
Ultimately, we are interested in the changes in asset prices,
source of potential \textit{returns}.
The changes in expectations about 
future cash flows or discount rate. 
So,
we are not so much interested in the level of
expectations of climate risks, but in the changes
in beliefs about climate risks.

Investors form expectations about climate risks
in an horizon $h$ with information available (those 
are conditional expectations).
Each period, new information arrives and investors
update their beliefs about climate risks:

\begin{equation}
    \Delta E_t (CC_{t+h}) = E_t(CC_{t+h}) - E_{t-1}(CC_{t+h})
\end{equation}

with $\Delta E_t (CC_{t+h})$ the \textit{innovation} or 
\textit{news} in climate risk. On the other hand, we have the \textit{unexpected} returns $\tilde{R}_t$:

\begin{equation}
    \tilde{R}_t = R_t - E_{t-1}(R_t)
\end{equation}

with $R_t$ the \textit{realized} returns 
and $E_{t-1}(R_t)$ the expected returns based on information available at time $t-1$.
Because investors reprice assets based on the arrival of new information,
we can expect that the innovation in climate risk is reflected in the unexpected returns:

\begin{equation}
    \tilde{R}_t = \beta \Delta E_t(CC_{t+h}) + \varepsilon_t
\end{equation}


with $\beta$ non-null as long as we expect 
that changes in climate risks affect investors 
expectations about future cash flows or discount rate.


\section{Measuring Climate Risks}

We have seen that what matters for asset prices
are expectations about climate risks. What 
matters about returns are the changes in
expectations about climate risks.

For more traditional macroeconomic factors, 
creating a time series that capture expectations 
of these factors is not so difficult. You may 
use data from the central bank, the government,
\textit{etc}. They publish leading indicators,
surveys, \textit{etc}. The task is more 
challenging for climate risks.

A common approach in the literature (see Engle et al. (2020) \cite{engle2020hedging})
is to use newspapers coverage of climate events as a proxy 
for the average investor's beliefs about climate risks.
As they noted, when there are events that plausibly
contains information about changes in climate risk,
this will likely leads to newspaper coverage of these events.
Newspapers may even be the direct source that investors
use to update their beliefs about climate risk.
Following the initial work from Engle et al. (2020) \cite{engle2020hedging},
researchers have developed a number of climate 
news series series capturing a variety of different climate risks.
Each measure is signed such that a large number 
corresponds to "bad news".




\section{Tag Index}


\section{Similarity Index}

\section{Conclusion}

In what follow we propose a method to construct
hedge portfolios with tradable assets that
mimic the behavior of climate risks. 
