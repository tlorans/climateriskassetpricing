\chapter{Economic Tracking Portfolio for News}

\section{Basics}

\subsection{Time Series}

\section{Tracking the Unexpected}

it constructs tracking portfolios for future (not current) 
economic variables, since asset returns reflect 
information about future cash flows and discount
rates. 
Second, as a consequence, it uses 
only the unexpected component of
returns (not total returns) 
in constructingthe tracking portfolios.

\subsection{Returns Predictability}

Suppose expected returns $x_t = E_t(R_{t+1}$ follow an 
AR(1) process:

\begin{equation}
    x_t = \phi x_{t-1} + \epsilon^x_t
\end{equation}

\begin{equation}
    R_{t+1} = x_t + \epsilon^R_{t+1}
\end{equation}


\subsection{Return Identity}

If an asset lasts one period, returns are, by definition:

\begin{equation}
    R_{t+1} = \frac{D_{t+1} }{P_t}
\end{equation}

where $D_{t+1}$ is the dividend paid at $t+1$ and $P_t$ is the price at $t$.
We don't have $P_{t+1}$ in this one period case because the asset 
lasts only one period.

We can take expectation:

\begin{equation}
    E_t(R_{t+1}) = \frac{E_t(D_{t+1}) }{P_t}
\end{equation}

and solve for price or price-dividend ratio:

\begin{equation}
    P_t = \frac{E_t(D_{t+1})}{E_t(R_{t+1})}
\end{equation}

\begin{equation}
    \frac{P_t}{D_t} = \frac{E_t(D_{t+1}/D_t)}{E_t(R_{t+1})}
\end{equation}

These formula represent the price as the discounted value 
of future dividends. If expected future dividends are higher,
the price goes up. If expected returns rise, the price goes down.
Why? Perceived risk goes up, so investors try to sell the asset,
which lowers the price. So, the price or the price-dividend ratio
should move only if expected returns or expected dividends move.
That is, if the discount rate or expected future cash flows change.

From here, we can reformulate this in terms of returns. We
can take surprises $\Delta E_{t+1}$ for the price:

\begin{equation}
    \Delta E_{t+1} (P_t) = \frac{\Delta E_{t+1}(D_{t+1})}{\Delta E_{t+1}(R_{t+1})}
\end{equation}

with $\Delta E_{t+1}(P_t) = 0$ because the price is known at $t$.


\subsubsection{Time-Varying Expected Returns}

To see if returns are predictable, we run a regression:

\begin{equation}
    R_{t+1} = a + bx_t + \epsilon_{t+1}
\end{equation}

This is the regression of tomorrow's return on today's
information $x_t$. If we find a big $b$ or large $R^2$,
we can say that returns are (somewhat) predictable.
If not, we can say that returns are not predictable.
The regression answer the question: can we predict
returns using today's information?

Equivalently, the forecasting regression implies:

\begin{equation}
    E_t(R_{t+1}) = a + bx_t
\end{equation}

This is the expectation of tomorrow's return given today's
information. The expected return can vary over time, 
being higher on low depending on the signal $x_t$.
Therefore, the forecasting regression measures whether 
expected returns vary over time.

\textit{Predictable} doesn't mean perfectly. Expected means 
conditional mean, but there is a lot of variance. Risk 
includes unexpected positive returns.

Use forecasting regression in finance to understand 
how the right hand variable is formed, from expectations 
of the left hand variable. 

If we want to check a forecaster, we run:

\begin{equation}
    \text{temperature}_{t+1} = a + b \times \text{forecast made at t} + \epsilon_{t+1}
\end{equation}

In terms of \textit{causality}, we think the forecaster 
gets information about future temperature, this causes him to 
issue a forecast. If it's a good forecast, then $b=1$ with a good
$R^2$. What we learn is not what causes the weather to be good 
or bad but how the forecast is formed.



\subsection{Tracking Portfolios}

This paper constructs portfolios with unexpected returns 
that are maximally correlated with unexpected components 
of future $y$. Specifically, the target variable is \textit{news}
about $y_{t+k}$, where $y_{t+k}$ is a macroeconomic variable
in period $t+k$. 

News is innovation in expectations about $y_{t+k}$, with 
notations:
\begin{equation}
    \Delta E_t(y_{t+k}) = E_t(y_{t+k}) - E_{t-1}(y_{t+k})
\end{equation}

For example, $\Delta E_t(y_{t+k})$ can be the news the market 
learns in July 2021 about the GDP growth rate in 2022.

The tracking portfolio returns are: 
\begin{equation}
    r_{t-1,t} = bR_{t-1,t}
\end{equation}

where $r_{t-1,t}$ is the return of the tracking portfolio,
$R{t-1,t}$ is a column vector of asset returns from the end 
of period $t-1$ to the end of period $t$, and $b$ is a row 
vector of portfolio weights.

The tracking portfolio is constructed using unexpected returns 
on the base assets. The unexpected returns are actual returns 
minus expected returns:

\begin{equation}
    \tilde{R}_{t-1,t} = R_{t-1,t} - E_{t-1}(R_{t-1,t})
\end{equation}

The portfolio weights are chosen so that $\tilde{r}_{t-1,t}$
is maximally correlated with $\Delta E_t(y_{t+k})$.

Estimating tracking portfolios for news is only slightly 
more complicated than estimating simple tracking portfolios. 
One can always write a projection equation of news on 
unexpected returns. The key assumption in this paper 
is that innovations in returns reflect innovations in 
expectations about future variables, so that the vector $a$
has non-zero elements in the projection equation:

\begin{equation}
    \Delta E_t(y_{t+k}) = a\tilde{R}_{t-1,t} + \eta_t
\end{equation}

where $\eta_t$ is the component of news that is orthogonal
to the unexpected returns. Since unexpected returns reflect
news about future cash flows and discount rates, $a$
will generally be non-zero for any variable correlated 
with future cash flows or discount rates (there is no 
intercept in the equation above).

It seems at first glance that one needs to obtain $\Delta E_t(y_{t+k})$,
the period $t$ news about $y_{t+k}$, to run the regression.
In fact, all that is needed is $\tilde{R}_{t-1,t}$, the
unexpected returns. The realization of $y_{t+k}$ 
can be written as the sum of the expectation in period $t-1$,
the innovation in expectations in period $t$, and the
innovation in expectations from period $t$ to period $t+k$:

\begin{equation}
    y_{t+k} = E_{t}(y_{t+k}) + e_{t,t+k} = E_{t-1}(y_{t+k}) + \Delta E_t(y_{t+k}) + e_{t,t+k}
\end{equation}

The second assumption made here is that expected returns 
on the base assets in period $t$ are linear functions 
of $Z_{t-1}$, a vector of control variables known at 
period $t-1$:

\begin{equation}
    E_t(R_{t-1,t}) = d Z_{t-1}
\end{equation}

While this assumption is a potential source of model 
misspecification, one might expect the empirical results 
to be relatively robust to this form of misspecification, 
since asset returns are largely unpredictable at short 
horizons.

Last, for notational convenience, define the projection
equation of lagged expectations of $y$ on the lagged control 
variables:

\begin{equation}
    E_{t-1}(y_{t+k}) = f Z_{t-1} + \mu_{t-1}
\end{equation}

Combining the last three equations, we obtain:

\begin{equation}
    y_{t+k} = b R_{t-1,t} + c Z_{t-1} + \epsilon_{t,t+k}
\end{equation}

where $b = a$, $c = f - ad$, and $\epsilon_{t,t+k} = e_{t,t+k} + \eta_t + \mu_{t-1}$.
This is a regression equation with realized future $y$ on the 
left-hand side and period $t$ returns and period $t-1$ control
variables in the right-hand side. It is consistent because 
the three components of $\epsilon_{t,t+k}$ are orthogonal to
both $R_{t-1,t}$ and $Z_{t-1}$.

This regression produces $bR_{t-1,t}$, the portfolio returns 
having unexpected components maximally correlated with 
$\Delta E_t(y_{t+k})$. The portfolio weights are $b$.
This equation is atheoretical and depends only on the 
assumption that changes in expectations about future $y$ 
are reflected in asset returns, and that expected asset 
returns are a function of the lagged control variables.





\section{Python Project: Tracking Inflation}

\section{Conclusion}