\chapter{Climate Risk Mimicking Portfolio}

How investors can mitigate the risks that climate 
change poses to their portfolios is a pressing question.
This is particularly important since many 
of the effects of climate change are sufficiently far 
in the future that neither derivatives or specialized 
insurance markets are available to directly hedge them. 
Instead, investors are largely forced to insure against 
realizations of climate risk by building hedge 
portfolios on their own.

Engle \textit{et al.} (2020) propose an approach 
to hedging climate risk, using climate news and the mimicking portfolio
approach, following 
the methodology we have seen in the previous chapter. 
The key question is how to measure news in the case of climate
change? 

\section{Hedging Climate News}

The mimicking portfolio approach combines a pre-determined set of 
assets into a portfolio that is maximally correlated with a given 
climate change shock, using historical data. To obtain the mimicking
portfolios, we estimate the following regression model:

\begin{equation}
    \label{eq:regression}
    CC_t = w R_t + \epsilon_t
\end{equation}

where $CC_t$ denotes the (mean zero) climate hedge target in month $t$,
$w$ is a vector of $N$ portfolio weights, $R_t$ is the $N \times 1$ vector 
of unexpected excess returns and $\epsilon_t$ is the regression residual.
The portfolio weights are estimated each month using a rolling window of
$T$ months of historical data. 

The approach is intuitive: the portfolio overweights assets that have
rise in value in the arriveal of climate change news, and underweights
assets that fall in value in the same situation. 
In doing so, the hedge portfolio profits when adverse climate 
change news occurs.


\section{Python Project: Transition Risk Hedging}

\section{Conclusion}