\chapter{Climate Transition Risk and Asset Pricing}

Uncertainty about states of world with climate risks 
is a key driver of asset prices. Uncertainty
about the states of the world with climate risks
can be thought with the uncertainty about the 
climate scenarios.


\section{Climate Transition Risk Dynamic}

The evolution of aggregate consumption growth $\Delta c_{t+1}$
is described as:

\begin{equation}
\Delta c_{t+1} = \mu + x_t + J_{t+1}
\end{equation}

where $\mu$ is the unconditional average, $x_t$ represents the
time-varying expected consumption growth and 
$J_{t+1}$ is the climate transition shock to the economy.
It can be referred to the abrupt introduction of 
a carbon tax or a sudden shift in climate policy.


\begin{tcolorbox}[colback=white, colframe=black, title=Example X]
    Sample Paths: Trend Growth and Above-Trend Growth
\end{tcolorbox}

The expected consumption growth is:

\begin{equation}
    x_{t+1} = \mu_x + \rho x_t + \phi J_{t+1}
\end{equation}

with a persistence parameter $\rho$ and innovations also driven 
by the shock $J_{t+1}$. The parameter $\phi$ captures the way 
the climate shock $J_{t+1}$ affects future path of consumption
growth.

\begin{tcolorbox}[colback=white, colframe=black, title=Example X]
    Sample Paths: Above-Trend Growth,With andWithout a Climate 
    Transition Shock
\end{tcolorbox}

In any period, the transition shock $J_{t+1}$ can take the value 
$-\varepsilon \in (0,1)$ with probability $\lambda_t$ or $0$ with 
probability $(1 - \lambda_t)$. The probability $\lambda_t$ is
itself time-varying, with the dynamics:

\begin{equation}
    \lambda_{t+1} = \mu_{\lambda} + \alpha \lambda_t 
    + \eta x_t + \xi J_{t+1}
\end{equation}


in addition to an autoregressive term, 
it depends on lagged measures of economic activity ($x_t$)
as well as on the current shock $J_{t+1}$.


\begin{tcolorbox}[colback=white, colframe=black, title=Example X]
    Time-varying transition risk
\end{tcolorbox}

We think about climate transition risk in terms 
of a relatively low-probability catastrophic event 
that could dramatically impact the economy. 

The parameter $\lambda_t$ captures the conditional
probability of such an abrupt shift occurring and 
$\varepsilon$ captures the magnitude of the shock.

The occurence of the abrupt low-carbon shift also affects the 
future path of the economy, since it directly affects 
the expected consumption growth $x_t$. 
Specifically, when $\phi > 0$, the transition shock
reduces not only consumption immediately 
but also future expected consumption growth. 

\begin{tcolorbox}[colback=white, colframe=black, title=Example X]
    xxx
\end{tcolorbox}

When instead, $\phi < 0$, there is a partial mean reversion
after a transition shock. This case has an especially 
interesting interpretation when modelling climate transition:
it captures the ability of the economy to shift towards 
a low-carbon economy.

\begin{tcolorbox}[colback=white, colframe=black, title=Example X]
    xxx
\end{tcolorbox}

Through the parameter $\xi$, a transition 
realization also affects future climate transition 
risk $\lambda_{t}$.

\begin{tcolorbox}[colback=white, colframe=black, title=Example X]
    xxx
\end{tcolorbox}

Finally, the model allows for feedback effects between 
climate change and the economy. Climate risks affect 
the economy (when the climate disaster occurs) and 
the economy affects climate risks through the effects 
of $x_t$ on $\lambda_{t+1}$ (modulated through the parameter 
$\eta$): when economic activity is high, climate risk 
increases. But when the climate shock materializes, 
consumption is low. 

\begin{tcolorbox}[colback=white, colframe=black, title=Example X]
    xxx
\end{tcolorbox}


\section{Climate Transition Risk Valuation}

Once the physical process of climate change is modeled,
one need to choose an utility function for the representative 
investor. It implies a stochastic discount factor (SDF)
that prices assets in the economy, at equilibrium. 

When the uncertainty emanates directly from the climate process 
itself, climate damage tends to be unexpectedly high in times 
when consumption is low because climate disaster realization 
are a primary driver of reduced consumption. 

\section{Climate Transition Risk Beta and Excess Returns}

Assets that are positively exposed to climate risk - that is, 
assets with low payoffs when climate damages are high -
thus tend to require positive risk premia. On the other hand, 
assets that are negatively exposed to climate risk - 
assets that payoffs primarily when climate are realized - 
tend to require negative risk premia since these assets 
provide an insurance against bad (high marginal utility)
states of the world. 

DERIVATION OF BETAS

\begin{tcolorbox}[colback=white, colframe=black, title=Example X]
    xxx
\end{tcolorbox}


\section{Conclusion}

We've focused on climate transition risk, but same approach 
can be applied to physical risk. 

%\begin{figure}[htbp]
    %     \centering
    % \begin{tikzpicture}[->, thick, auto, node distance=4cm]
    %     % Nodes
    %     \node[circle, draw] (current) at (0, 0) {CP};
    %     \node[circle, draw] (net_zero) at (4, 2) {NZ};
    %     \node[circle, draw] (no_transition) at (4, -2) {CP};
    
    %     % Edges
    %     \draw (current) -- node[midway, above] {Transition} (net_zero);
    %     \draw (current) -- node[midway, below] {No Transition} (no_transition);
    
    %     % Labels for time
    %     \node at (0, -1) {t};
    %     \node at (4, -3) {t+1};
    % \end{tikzpicture}
    %     \caption{Climate Transition Risk}
    %     \label{fig:climate_transition}
    % \end{figure}
    
    
    % \begin{figure}[htbp]
    %     \centering
    %     \begin{tikzpicture}[scale=1.5, every node/.style={font=\footnotesize}]
    
    %     % Nodes
    %     \node (start) at (0,0) [draw, circle] {CP};
    %     \node (current) at (2,1.5) [draw, circle] {CP};
    %     \node (transition) at (2,-1.5) [draw, circle] {NZ};
    %     \node (current_green) at (4,2.5) {Green asset: Low payoff};
    %     \node (current_brown) at (4,1) {Brown asset: High payoff};
    %     \node (transition_green) at (4,-1) {Green asset: High payoff};
    %     \node (transition_brown) at (4,-2.5) {Brown asset: Low payoff};
    
    %     % Edges
    %     \draw[->] (start) -- (current) node[midway, above] {No Transition};
    %     \draw[->] (start) -- (transition) node[midway, below] {Transition};
    %     \draw[->] (current) -- (current_green);
    %     \draw[->] (current) -- (current_brown);
    %     \draw[->] (transition) -- (transition_green);
    %     \draw[->] (transition) -- (transition_brown);
        
    %      % Labels for time
    %      \node at (0, -1) {t};
    %      \node at (4, -3) {t+1};   % Labels
    
    % \end{tikzpicture}
    %     \caption{Climate States of the World}
    %     \label{fig:climate_risk}
    % \end{figure}
    