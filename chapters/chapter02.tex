\chapter{Climate Risk Hedging Porfolios}
 



\section{Mimicking Approach}


Ross (1976) \cite{ross1976apt} introduced 
the concept of \textit{arbitrage pricing theory} (APT).
In this model, the expected return of an asset is
a linear function of a set of risk factors. Famous examples of risk factors are the
\textit{Fama-French factors} (see Fama and French (1993) \cite{fama1993common}).
Those factors are the excess return of the market,
the excess return of small cap stocks over big cap stocks
and the excess return of high book-to-market stocks over low book-to-market stocks:

\begin{equation}
    E(R_i) = \beta_m R_m + \beta_{smb} R_{smb} + \beta_{hml} R_{hml}
\end{equation}

with $E(R_i)$ the expected return of asset $i$,
$R_m$ the excess return of the market, $R_{smb}$ the excess return of small cap stocks over big cap stocks,
$R_{hml}$ the excess return of high book-to-market stocks over low book-to-market stocks,
$\beta_m$ the market beta of asset $i$, $\beta_{smb}$ the size beta of asset $i$ and $\beta_{hml}$ the value beta of asset $i$.
Those factors are tradable, as 
they are directly traded in financial markets (you can buy 
the market, small cap stocks and high book-to-market stocks
and short sell the opposite side of the trade).

Macroeconomic factors are examples of 
\textit{non-tradable factors} (think about inflation, 
industrial growth, \textit{etc}). Economic conditions 
have pervasive effects on asset returns (see Flannery 
and Protopapadakis (2002) \cite{flannery2002macroeconomic}).
A standard way to tackle the problem of non-tradable factors
is to use factor mimicking portfolios (FMPs), such 
as in Jurczenko and Teiletche (2022) \cite{jurczenko2022macro}. That is,
to construct a portfolio of tradable assets that
mimics the behavior of non-tradable factors. 


Climate risks are non-tradable factors,
as they are not directly traded in financial markets
(see Jurczenko and Teiletche (2023) \cite{jurczenko2023climate}).
We can use the same approach of FMPs to construct
a portfolio of tradable assets that mimics the behavior
of climate risks.

\begin{equation}
\Delta E_t(CC_{t+h}) = w^T \tilde{R}_{t} + \varepsilon_t
\end{equation}

FIGURE 2 IN JURCENZKO MACRO FACTORS WITH THIS METHOD
ML macro FMPs vs underlying macro factors
\section{Narrative Approach}

FIGURE 2 IN JURCENZKO MACRO FACTORS WITH THIS METHOD
ML macro FMPs vs underlying macro factors

\section{Risk Premia}

Problem with short time series to infer risk premia

\section{Conclusion}

