\chapter{Factor Signal}

\section{Regression and Few Statistics}

We will run \textit{regression} to estimate the 
\textit{beta coefficient}. For example, we can 
regress a return on the market return:

\begin{equation}
    R_{t} = \alpha + \beta R_{m,t} + \epsilon_{t}
\end{equation}

We can also do it with the returns of several other portfolios:

\begin{equation}
    R_{t} = \alpha + \beta_m R_{m,t} + \gamma R_{p,t} + \epsilon_{t}
\end{equation}

The \textit{population value} of the beta coefficient with the single factor model is:

\begin{equation}
    \beta = \frac{Cov(R,R_m)}{Var(R_m)}
\end{equation}

The regression recovers the \textit{true} (\textit{i.e.} unbiased) 
$\beta$ only if the \textit{error term} $\epsilon$ is \textit{uncorrelated}
with the right hand variables ($R_m$ in our first case).

\begin{examplebox}
    \textbf{Example X.1.} 
    \textit{example that illustrates the idea
    that the beta coefficient is unbiased if 
    error term is uncorrelated with the right hand variables.}
\end{examplebox}

In the multiple regression example, 
$\beta_m$ captures the effect on the return of only movements in the market portfolio 
that are not correlated with movement in the other portfolio $p$.

\begin{examplebox}
\textbf{Example X.1.} 
\textit{example that illustrates the idea 
that in multiple regression, 
the beta coefficient captures the effect of the
market return on the portfolio return that is not
explained by the other portfolio.}
\end{examplebox}

\section{Factor Mimicking Portfolio}

\textit{Factor mimicking} is a useful practice in finance. 
The core idea is to replace some variables with 
a linear combination of other variables. More
specifically, \textbf{some variable of interest can be
written as a portfolio of \textit{tradable assets}}.
It can be used to proxy economic variable that 
are not directly observable with tradable assets.

This is the idea from Lamont (2001): 
we can construct, from financial assets, 
a "matching portfolio" of some economic factor
that is not directly observable (in the case of 
Lamont, the idea was to forecast variables).



\begin{examplebox}
    \textbf{Example X.1.}    
    Say you want to estimate current (this is \textit{nowcasting})
    GDP or inflation. You can construct the portfolio 
    of assets that best mimics the movements of GDP or inflation.
    Once you've run your regression, you can use your 
    estimate of returns to predict the macro variable.
    Let's say we don't have individual 
    stock returns and we want to estimate 
    the market return. All what we have is
    the return of $K$ industry portfolios. We 
    can estimate the market return as a 
    linear combination of the industry returns:
    
    \begin{equation}
    R_{m,t} = \beta_{\text{energy}} R_{\text{energy},t} + \beta_{\text{financials}} R_{\text{financials},t} + \ldots + \beta_{k} R_{k,t} + \epsilon_{t}
    \end{equation}
    
    It looks very much like a portfolio, with 
    the estimated $\beta$ as the weights of the assets.
\end{examplebox}



Now, let's say we want to form a portfolio that 
mimics the behavior of a \textit{factor signal} $y$.
Specifically, our target is "news" or \textit{innovations} about the signal,
defined as the difference between the current expectation 
and the previous expectation:

\begin{equation}
    \Delta E_t(y_{t+1}) := E_t(y_{t+1}) - E_{t-1}(y_{t+1})
\end{equation}

In can be for example the news that the market 
learns about the industrial production in May about 
the industrial production in June.

The mimicking portfolio portfolio returns are:

\begin{equation}
    R_{y, t} = w^T R_t
\end{equation}

where $R_t$ is the vector of returns of the tradable assets.
You construct the mimicking portfolio with 
\textit{unexpected returns} of the tradable assets. Unexpected 
returns are actual returns minus expected returns:

\begin{equation}
    \tilde{R_t} := R_t - E_{t-1}(R_t)
\end{equation}

with the assumption that \textbf{the expected returns are a linear 
function of factors $F_{t}$}:

\begin{equation}
    E_{t-1}(R_t) = \gamma^T F_{t}
\end{equation}


The porfolio weights of the mimicking portfolio of $y$ 
are chosen so that $\tilde{R}_{y,t}$ is as close as possible
to $\Delta E_t(y_{t+1})$ (maximally correlated).

To do this, the key assumption is that 
\textbf{innovations in returns (unexpected returns)
reflect innovations in expectations about 
the factor signal}, such that:

\begin{equation}
    \Delta E_t(y_{t+1}) = \beta^T \tilde{R}_t + \epsilon_t
\end{equation}

If the factor signal $y$ is correlated 
with future cash flows and discount rates,
then we may find something in the $\beta$, relating
news reflected in the unexpected returns. 
Again, this is based on the assumption that 
the unexpected returns reflect news about the
future cash flows and discount rates (\textit{i.e.}
about $\Delta E_t(y_{t+1})$).

Recalling that the returns are $R_t = E_{t-1}(R_t) + \tilde{R}$,
we can therefore rewrite it with the factors:

\begin{equation}
    R_t = \gamma^T F_{t} + \tilde{R}
\end{equation}

and then includes the innovations in the factor signal:

\begin{equation}
    R_t = \gamma^T F_{t} + \eta \Delta E_t(y_{t+1}) + u_{t}
\end{equation}

What we have here? 
\textbf{The returns of any asset can be written 
as a function of its expected returns ($\gamma^T F_{t}$)
and the unexpected returns. The unexpected returns are 
decomposed into the news about the factor signal $\Delta E_t (y_{t+1})$ and
uncorrelated errors ($u_t$).}

\section{Climate Factor Signal}

In the case of climate risk, we have:

\begin{equation}
    \Delta E_t (CC_{t+k}) = E_t (CC_{t+k}) - E_{t-1} (CC_{t+k})
\end{equation}

with $CC_{t+k}$ the climate risk at an undefined horizon $k$.
We have seen in the chapter 1 how to use text to proxy for 
the market expectation of the climate risk $E_t(CC_{t+k})$.


\begin{examplebox}
    \textbf{Example X.1.}    
    \textit{illustrates use of demean vs. AR(1) to estimate}
\end{examplebox}


Therefore, we can estimate:

\begin{equation}
    R_{i,t} = \beta_i \Delta E_t(CC_{t+k}) + \gamma_i^T \text{Factors}_t + \epsilon_{i,t}
\end{equation}

The $\beta_i$ for each portfolio is the signal 
upon which we want to derive our mimicking portfolio.
The higher is $\beta_i$, the higher will be the weight 
of the portfolio in the mimicking portfolio.

However, if $\beta_i$ is negative, it means that
we should short the portfolio. In practice, 
it is less common to short portfolios, so we
can set $\beta_i = 0$ if it is negative. In that 
case, the mimicking portfolio would be a \textit{long-only}
portfolio.

A simple method to go from the $\beta_i$ to the
weights of the mimicking portfolio is to normalize it:
\begin{equation}
    w_i = \frac{\tilde{\beta_i}}{\sum_{i=1}^{N} \tilde{\beta_i}}
\end{equation}

with $\tilde{\beta_i} = \max(\beta_i, 0)$.

The vector of weights $w$ is already an intuitive 
long-only mimicking portfolio: it simply 
\textbf{weights the portfolios based on their positive 
$\beta_i$ on the climate innovation signal 
$\Delta E_t(CC_{t+k})$ we wish to mimic}.
 

\section{Further Reading}

\section{Exercises}

\section{Solutions}

\section{Python Project}