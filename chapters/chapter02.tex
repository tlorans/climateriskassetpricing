\chapter{Climate Risk Mimicking Porfolios}
 
Two main approaches of FMPs 
have been proposed in the literature: 
(i) the two-pass cross-sectional regression (Fama and MacBeth, 1973) and
(ii) the maximum correlation portfolio (MCP) (Huberman et al, 1987).


\section{Mimicking Climate News}

The vector of weights $w_k$ 
is the solution to the following optimization problem:

\begin{equation}
    \begin{aligned}
        & \underset{w_k}{\min}
        & &  \frac{1}{2} w_k^T \Sigma w_k \\
        & \text{subject to}
        & & B^T w_k = \beta_k
    \end{aligned}
\end{equation}

where $B$ is the $N \times K$ matrix of factor loadings, $\beta_k$ is the $K \times 1$ vector of factor exposures,
with the $k$-th element equal to 1 and the other elements equal to $\beta_{k,l}$, 
and $\Sigma$ is the $N \times N$ covariance matrix of asset returns.

We can form the Lagrangian:

\begin{equation}
    \mathcal{L}(w_k, \lambda) = \frac{1}{2} w_k^T \Sigma w_k - \lambda_k^T (B^T w_k - \beta_k)
\end{equation}

where $\lambda_k$ is the $K \times 1$ vector of Lagrange multipliers.

The first order condition is:

\begin{equation}
    \begin{aligned}
        \frac{\partial \mathcal{L}}{\partial w_k} &= \Sigma w_k - B \lambda = 0 \\
        \Rightarrow w_k &= \Sigma^{-1} B \lambda_k
    \end{aligned}
\end{equation}

Substituting $w_k$ in the constraint, we have:

\begin{equation}
    \begin{aligned}
        B^T w_k &= \beta_k \\
        B^T \Sigma^{-1} B \lambda_k &= \beta_k \\
        \Rightarrow \lambda_k &= (B^T \Sigma^{-1} B)^{-1} \beta_k
    \end{aligned}
\end{equation}

Substituting $\lambda_k$ in $w_k$, we finally have 
the solution to the optimization problem:

\begin{equation}
    w_k^* = \Sigma^{-1} B (B^T \Sigma^{-1} B)^{-1} \beta_k
\end{equation}

Taking together all the $K$ factors, we have the matrix of weights $W$:

\begin{equation}
    W = \Sigma^{-1} B (B^T \Sigma^{-1} B)^{-1} B_K
\end{equation}

where $B_K$ is the $K \times K$ matrix with the $k$-th column equal to $\beta_k$ and the other columns equal to $\beta_{k,l}$.


\section{Two-Pass Fama-MacBeth}

In the case of the two-pass Fama-MacBeth,
assets are uncorrelated and have constant variance.

\begin{equation}
        \Sigma = \sigma^2 I_N \\
\end{equation}

where $\sigma^2$ is the variance of the asset returns.

$B$ is multivariate (i.e., $K > 1$) and the 
target exposure is:

\begin{equation}
    B_K = I_K 
\end{equation}

That is, we have a $beta$ of one to the $k$-th factor and 
zero to the others.

Substituting $\Sigma$ and $B_K$ in the equation (3.1), we have:


WHY $\sigma^2 I_N$ and $I_K$ cancels out?

\begin{equation}
    \begin{aligned}
        W &= {\sigma^2}I_N B (B^T B)^{-1} I_K \\
        &= B (B^T B)^{-1}
    \end{aligned}
\end{equation}


FMP composition as estimated by different methods

FIGURE 2 IN JURCENZKO MACRO FACTORS WITH THIS METHOD

\section{Maximum Correlation Portfolio}

We have the Target-Beta MCP, where 
$B$ is univariate (i.e., $K = 1$) and the
target exposure is:

\begin{equation}
    B_K = B^T \Sigma^{-1} B
\end{equation}

Substituting $B_K$ in the equation (3.1), we have:

FIND THE INTERMEDIARY STEPS
\begin{equation}
    \begin{aligned}
        W &= \Sigma^{-1} B (B^T \Sigma^{-1} B)^{-1} B^T \Sigma^{-1} B \\
        &= \Sigma^{-1} B
    \end{aligned}
\end{equation}


FMP composition as estimated by different methods

FIGURE 2 IN JURCENZKO MACRO FACTORS WITH THIS METHOD

\section{Other Approaches}

Narrative approach