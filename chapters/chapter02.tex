\chapter{Climate Risk Mimicking Porfolios}
 
Two main approaches of FMPs 
have been proposed in the literature: 
(i) the two-pass cross-sectional regression (Fama and MacBeth, 1973) and
(ii) the maximum correlation portfolio (MCP) (Huberman et al, 1987).

It is possible to recover both approaches
with the equation in the chapter 1:

\begin{equation}
    W = \Sigma^{-1} B (B^T \Sigma^{-1} B)^{-1} B_K
\end{equation}


\section{Two-Pass Fama-MacBeth}

In the case of the two-pass Fama-MacBeth,
assets are uncorrelated and have constant variance.

\begin{equation}
        \Sigma = \sigma^2 I_N \\
\end{equation}

where $\sigma^2$ is the variance of the asset returns.

$B$ is multivariate (i.e., $K > 1$) and the 
target exposure is:

\begin{equation}
    B_K = I_K 
\end{equation}

That is, we have a $beta$ of one to the $k$-th factor and 
zero to the others.

Substituting $\Sigma$ and $B_K$ in the equation (3.1), we have:


WHY $\sigma^2 I_N$ and $I_K$ cancels out?

\begin{equation}
    \begin{aligned}
        W &= {\sigma^2}I_N B (B^T B)^{-1} I_K \\
        &= B (B^T B)^{-1}
    \end{aligned}
\end{equation}


FMP composition as estimated by different methods

FIGURE 2 IN JURCENZKO MACRO FACTORS WITH THIS METHOD

\section{Maximum Correlation Portfolio}

We have the Target-Beta MCP, where 
$B$ is univariate (i.e., $K = 1$) and the
target exposure is:

\begin{equation}
    B_K = B^T \Sigma^{-1} B
\end{equation}

Substituting $B_K$ in the equation (3.1), we have:

FIND THE INTERMEDIARY STEPS
\begin{equation}
    \begin{aligned}
        W &= \Sigma^{-1} B (B^T \Sigma^{-1} B)^{-1} B^T \Sigma^{-1} B \\
        &= \Sigma^{-1} B
    \end{aligned}
\end{equation}


FMP composition as estimated by different methods

FIGURE 2 IN JURCENZKO MACRO FACTORS WITH THIS METHOD