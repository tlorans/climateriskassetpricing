\chapter{ESG Preferences}



\section{Expected Utility and Optimal Portfolio}


\subsection{Setting the Investor's Expected Utility}

Let's assume a single period model, from $t=0$ to $t=1$.
We have $N$ stocks. 

We have a $N \times 1$ vector of returns $\tilde{r}_1$ at period 1, 
assumed to be normally distributed: 

\begin{equation}
    \tilde{r}_1 = \mu + \tilde{\epsilon}_1
\end{equation}

with $\mu$ the equilibrium expected excess returns and $\tilde{\epsilon}_1$ the
random component of the returns $\tilde{\epsilon}_1 \sim N(0, \Sigma)$.

The investor $i$ has an exponential CARA utility function, with 
$\tilde{W}_{1,i}$ the wealth at period 1, and $X_i$ the
$N \times 1$ vector of portfolio weights.

\begin{equation}
    V(\tilde{W}_{1,i}, X_i) = -\exp{(-A_i \tilde{W}_{1,i}-b_i^T X_i)}
\end{equation}

with $A_i$ agent's absolute risk aversion, $b_i$ an $N \times 1$ vector of nonpecuniary 
benefits. 

\begin{equation}
    b_i = d_i g 
\end{equation}

with $g$ an $N \times 1$ vector and $d_i \geq 0$ a scalar measuring 
the agent's taste for the nonpecuniary benefits.


The expectation of agent $i$'s in period 0 are:

\begin{equation}
    E_0(V(\tilde{W}_{1,i}, X_i)) = E_0(-\exp{(-A_i \tilde{W}_{1,i}-b_i^T X_i)})
\end{equation}

We can replace $\tilde{W}_{1,i}$ by the relation 
$\tilde{W}_{1,i} = W_{0,i}(1 + r_f + X_i^T \tilde{r}_1)$
 and define $a_i := A_i W_{0,i}$. The idea is to make out from the
expectation the terms that we know about (in period 0), and reexpress 
the terms within the expectation as a function of the
portfolio weights $X_i$. The last two steps use the fact that 
$\tilde{r}_1$ is normally distributed with mean $\mu$ and variance $\Sigma$.

\begin{equation}
    \begin{aligned}
    E_0(V(\tilde{W}_{1,i}, X_i)) = E_0(-\exp{(-A_i W_{0,i}(1 + r_f + X_i^T \tilde{r}_1)-b_i^T X_i)}) \\
    = E_0(-\exp{(-a_i(1 + r_f + X_i^T \tilde{r}_1)-b_i^T X_i)})  \\
    = E_0(-\exp{(-a_i(1 + r_f) - a_i X_i^T \tilde{r}_1 - b_i^T X_i)}) \\
    = -\exp{(-a_i(1 + r_f))} E_0(-\exp{(-a_i X_i^T \tilde{r}_1 - b_i^T X_i)}) \\
    = -\exp{(-a_i(1 + r_f))} E_0(-\exp{(-a_i X_i^T (\tilde{r}_1 + \frac{b_i}{a_i}))})  \\
    = -\exp{(-a_i (1 + r_f))} \exp{(-a_i X_i^T (E_0(\tilde{r}_1) + \frac{b_i}{a_i})+\frac{1}{2}a_i^2 X_i^T \text{Var}(\tilde{r}_1)X_i)} \\
    = -\exp{(-a_i (1 + r_f))} \exp{(-a_i X_i^T (\mu + \frac{b_i}{a_i})+\frac{1}{2}a_i^2 X_i^T \Sigma X_i)} \\
    \end{aligned}
\end{equation}

\subsection{Solving for the Investor's Optimal Portfolio}

The investors choose their optimal portfolios at time 0. 
The optimal portfolio $X_i$ is the one that maximizes the expected utility.
To find it, we differentiate the expected utility with respect to $X_i$ and 
set it to zero, to obtain the first-order condition.

We are going to do it step by step:

\begin{enumerate}
    \item Combine the Exponential Terms:
    \begin{equation}
        E_0(V(\tilde{W}_{1,i}, X_i)) = -\exp{(-a_i(1 + r_f)-a_iX_i^T(\mu + \frac{b_i}{a_i})+\frac{1}{2}a_i^2X_i^T\Sigma X_i)}
    \end{equation}
    and let $f(X_i)$ be the exponent:
    \begin{equation}
        E_0(V(\tilde{W}_{1,i}, X_i)) = -\exp{f(X_i)}
    \end{equation}
    \item Differentiate $f(X_i)$ with respect to $X_i$. We have the 
    chain rule:
    \begin{equation}
        \frac{\partial h}{\partial X_i} = \frac{\partial h}{\partial f} \frac{\partial f}{\partial X_i}
    \end{equation}
    If $h = - \exp{(f)}$, then $\frac{\partial h}{\partial f} = - \exp{(f)}$. 
    Therefore we have:
    \begin{equation}
        \frac{\partial h}{\partial X_i} = -\exp{(f)} \frac{\partial f}{\partial X_i}
    \end{equation}
    To tackle the derivative of $f(X_i)$, we use two rules. First $\frac{\partial x^T b}{\partial x} = b$ and
    $\frac{\partial x^T A x}{\partial x} = 2Ax$ if $A$ is symmetric. We have:
    \begin{equation}
        \frac{\partial f}{\partial X_i} = -a_i(\mu + \frac{b_i}{a_i}) + a_i^2 \Sigma X_i
    \end{equation}
    Combining:
    \begin{equation}
        \frac{\partial h}{\partial X_i} = -\exp{(f)} ( -a_i(\mu + \frac{b_i}{a_i}) + a_i^2 \Sigma X_i)
    \end{equation}
    \item Set the derivative to zero:
    \begin{equation}
        \begin{aligned}
            -\exp{(f)} ( -a_i(\mu + \frac{b_i}{a_i}) + a_i^2 \Sigma X_i) = 0 \\
            -a_i(\mu + \frac{b_i}{a_i}) + a_i^2 \Sigma X_i = 0
        \end{aligned}
    \end{equation}
    where the exponential term is always positive, so we can drop it.
    \item Rearrange and solve for $X_i$:
    \begin{equation}
        \begin{aligned}
            a_i^2 \Sigma X_i = a_i(\mu + \frac{b_i}{a_i}) \\
            a_i \Sigma X_i = \mu + \frac{b_i}{a_i} \\
            \Sigma X_i = \frac{1}{a_i}( \mu + \frac{b_i}{a_i}) \\
            X_i = \frac{1}{a_i} \Sigma^{-1}(\mu + \frac{b_i}{a_i})
        \end{aligned}
    \end{equation}
\end{enumerate}


For the sake of simplicity, we assume that $a_i = a$ for all investors.
We now have:

\begin{equation}
    \begin{aligned}
    X_i = \frac{1}{a} \Sigma^{-1}(\mu + \frac{b_i}{a}) \\
    = \frac{1}{a} \Sigma^{-1}(\mu + \frac{d_i}{a}g) \\
    \end{aligned}
\end{equation}

Therefore, the optimal portfolio differs across investors due to the
ESG characteristics $g$ of the stocks and the investors' taste for
nonpecuniary benefits $d_i$. 


\begin{figure}
    \centering
    PLACEHOLDER
    % \includegraphics[width=0.8\textwidth]{images/ESG_taste.png}
    \caption{Efficient Frontier with ESG Preferences}
    \label{fig:esg_taste}
\end{figure}

\section{Heterogeneous Investors and Expected Returns}


\subsection{Heterogeneous Market}

The $n$th element of investor $i$'s portfolio weight vector $X_i$ is:

\begin{equation}
    X_{i,n} = \frac{W_{0,i,n}}{W_{0,i}}
\end{equation}

with $W_{0,i,n}$ the wealth invested in stock $n$ by investor $i$ at time 0.

The total wealth invested in stock $n$ at time 0 is:

\begin{equation}
    W_{0,n} := \int_i W_{0,i,n} di
\end{equation}

The $n$th element of the market-weight vector $w_m$ is:

\begin{equation}
    \begin{aligned}
    w_{m,n} = \frac{W_{0,n}}{W_{0}} \\
    \end{aligned}
\end{equation}

We can now express $W_{0,n}$ in terms of individual investors' wealths
by using the definition of $W_{0,n}$:

\begin{equation}
    w_{m,n} = \frac{1}{W_0} \int_i W_{0,i,n} di
\end{equation}

We now that $W_{0,i,n} = W_{0,i} X_{i,n}$, so we can rewrite the equation:

\begin{equation}
    w_{m,n} = \frac{1}{W_0} \int_i W_{0,i} X_{i,n} di
\end{equation}

Defining $\omega_i = \frac{W_{0,i}}{W_0}$, we have:

\begin{equation}
    \begin{aligned}
    w_{m,n} = \int_i \frac{W_{0,i}}{W_0} X_{i,n} di\\
    = \int_i \omega_i X_{i,n} di
    \end{aligned}
\end{equation}

We can now plug in $X_i$ to obtain $w_m$ the vector 
of market weights: 

\begin{equation}
    \begin{aligned}
    w_{m} = \int_i \omega_i X_i di \\ 
     = \int_i \omega_i \frac{1}{a} \Sigma^{-1}(\mu + \frac{d_i}{a}g)_n di \\
     = \frac{1}{a} \sigma^{-1} \mu (\int_i \omega_i di) + \frac{1}{a^2} \Sigma^{-1} g (\int_i \omega_i d_i di)
    \end{aligned}
\end{equation}


We have $\int_i \omega_i di = 1$ and we define $\bar{d} := \int_i d_i di \geq 0$,
the wealth-weighted mean of ESG tastes $d_i$ across agents. Therefore:

\begin{equation}
    w_m = \frac{1}{a} \Sigma^{-1} \mu + \frac{1}{a^2} \Sigma^{-1} g \bar{d}
\end{equation}

\subsection{Expected Returns}

Starting from the the vector of market weights $w_m$, we now can 
solve for $\mu$ the vector of expected returns. We have:

\begin{equation}
    \begin{aligned}
    w_m = \frac{1}{a} \Sigma^{-1} \mu + \frac{1}{a^2} \Sigma^{-1} g \bar{d} \\
    a w_m = \Sigma^{-1} \mu + \frac{1}{a} \Sigma^{-1} g \bar{d} \\
    a w_m - \frac{1}{a} \Sigma^{-1} g \bar{d} = \Sigma^{-1} \mu \\
    \Sigma (a w_m - \frac{1}{a} \Sigma^{-1} g \bar{d}) = \mu \\ 
    \mu = a \Sigma w_m - \frac{1}{a} \Sigma \Sigma^{-1} g \bar{d} \\
    \mu = a \Sigma w_m - \frac{1}{a} g \bar{d}
    \end{aligned}
\end{equation}

Multiplying by $w_m$, we find the market equity premium $\mu_m = w_m^T \mu$:

\begin{equation}
    \begin{aligned}
    \mu_m = a w_m^T \Sigma w_m - \frac{\bar{d}}{a} w_m^T g  \\
    = a \sigma_m^2 - \frac{\bar{d}}{a} w_m^T g 
    \end{aligned}
\end{equation}

where $\sigma_m^2 = w_m^T \Sigma w_m$ is the market return variance. 

\begin{figure}
    \centering
    PLACEHOLDER
    % \includegraphics[width=0.8\textwidth]{images/ESG_taste.png}
    \caption{$\mu_m$ and $w_m^Tg$ relationship.}
    \label{fig:green_market}
\end{figure}

The equity premium $\mu_m$ depends on the average of ESG tastes, $\bar{d}$,
through the "greeness" of the market portfolio $w_m^T g$.
If the market is net green (i.e., $w_m^T g > 0$), then stronger 
ESG tastes (higher $\bar{d}$) lead to lower equity premium.

Conversely, if the market is net "brown" ($w_m^T g < 0$), then stronger
ESG tastes lead to higher equity premium as investors demand 
compensation for holding brown stocks.


\subsection{Expected Excess Returns}

\subsubsection{Average Expected Excess Returns}

For simplicity, we assume that the market portfolio is ESG-neutral:

\begin{equation}
    w_m^T g = 0
\end{equation}

which implies that the equity premium is:

\begin{equation}
    \mu_m = a \sigma_m^2
\end{equation}

that is, independent of the average ESG tastes $\bar{d}$.

From the last equation, we note that $a = \frac{\mu_m}{\sigma_m^2}$, then 
the expected excess returns can be reexpressed as: 

\begin{equation}
    \begin{aligned}
        \mu = a \Sigma w_m - \frac{1}{a} g \bar{d} \\
        = \frac{\mu_m}{\sigma_m^2} \Sigma w_m - \frac{1}{a} g \bar{d} \\
        = \mu_m \beta_m - \frac{1}{a} g \bar{d} \\
    \end{aligned}
\end{equation}

where we have used the fact that the 
vector of market betas is $\beta_m = \frac{\Sigma w_m}{\sigma_m^2}$.

This gives the first proposition of the model:

\paragraph{Proposition 1.}
\textit{Expected excess returns in equilibirium are given by:}

\begin{equation}
    \mu = \mu_m \beta_m - \frac{\bar{d}}{a} g 
\end{equation}

\textit{
The expected excess returns deviate from their CAPM values due to ESG 
tastes for holding green stocks.}

\paragraph{Corrolary 1.}
\textit{If $\bar{d} > 0$, the expected return on stock $n$ is decreasing in $g_n$}.

\textit{Given their ESG tastes, agents are willing to pay more for greener firms, then 
lowering the firms' expected returns.}

\paragraph{Corrolary 2.} \textit{Because the vector of stocks' CAPM alphas is defined as 
$\alpha := \mu - \mu_m \beta_m$, we have:}

\begin{equation}
    \alpha_n = - \frac{\bar{d}}{a} g_n
\end{equation}

\textit{If $\bar{d} > 0$, green stocks have negative alphas, and brown stocks 
have positive alphas. Greener stocks have lower alphas.}


\begin{figure}
    \centering
    PLACEHOLDER
    % \includegraphics[width=0.8\textwidth]{images/ESG_taste.png}
    \caption{$\alpha_n$ relationship with $g_n$}
    \label{fig:esg_returns}
\end{figure}

\subsubsection{Investor $i$'s Excess Returns Mean and Variance}

Investor $i$'s expected excess return is given by:
\begin{equation}
    \begin{aligned}
E(\tilde{r}_{1,i}) = X_i^T \mu \\ 
    \end{aligned}
\end{equation}

We know that $\mu = \mu_m \beta_m - \frac{\bar{d}}{a} g$ from the Proposition 1:

\begin{equation}
    \begin{aligned}
E(\tilde{r}_{1,i}) = X_i^T (\mu_m \beta_m - \frac{\bar{d}}{a} g) \\
    \end{aligned}
\end{equation}

We can express $X_i$ in terms of $w_m$ by susbtracting the expression $w_m$ 
from the expression of $X_i$. Recall the assumption that $a_i = a$ and distribute:

\begin{equation}
    \begin{aligned}
        E(\tilde{r}_{1,i}) = (w_m^T + \frac{1}{a} \Sigma^{-1} (\mu + \frac{d_i}{a}g) - \frac{1}{a} \Sigma^{-1} \mu - \frac{\bar{d}}{a^2} \Sigma^{-1} g) (\mu_m \beta_m - \frac{\bar{d}}{a} g) \\
        = (w_m^T + \frac{1}{a} \Sigma^{-1}\mu - \frac{1}{a} \Sigma^{-1} \mu + \frac{d_i}{a^2}\Sigma^{-1}g - \frac{\bar{d}}{a^2} \Sigma^{-1}g)(\mu_m \beta_m - \frac{\bar{d}}{a}g)\\
        = (w_m^T + \frac{d_i - \bar{d}}{a^2} \Sigma^{-1} g) (\mu_m \beta_m - \frac{\bar{d}}{a} g) \\
    \end{aligned}
\end{equation}

Rewriting $d_i - \bar{d} = \delta_i$, recalling that $\beta_m = (\frac{1}{\sigma^2_m})\Sigma w_m$ and distribute:

\begin{equation}
    \begin{aligned}
        E(\tilde{r}_{1,i}) = (w_m^T + \frac{\delta_i}{a^2} \Sigma^{-1} g) (\frac{\mu_m}{\sigma^2_m} \Sigma w_m - \frac{\bar{d}}{a}g) \\
= w_m^T \frac{\mu_m}{\sigma^2_m} \Sigma w_m - w_m^T \frac{\bar{d}}{a} g + \frac{\delta_i \mu_m}{a^2\sigma^2_m}\Sigma^{-1}\Sigma g^T w_m - \frac{\delta_i \bar{d}}{a^3} g^T \Sigma g \\
= w_m^T \frac{\mu_m}{\sigma^2_m} \Sigma w_m - w_m^T \frac{\bar{d}}{a} g + \frac{\delta_i \mu_m}{a^2\sigma^2_m} g^T w_m - \frac{\delta_i \bar{d}}{a^3} g^T \Sigma g\\
    \end{aligned}
\end{equation}

We now that $w_m^T \Sigma w_m = \sigma^2_m$, so we have:

\begin{equation}
    \begin{aligned}
        E(\tilde{r}_{1,i}) = \mu_m - w_m^T \frac{\bar{d}}{a}g + \frac{\delta_i \mu_m}{a^2\sigma^2_m} g^T w_m - \frac{\delta_i \bar{d}}{a^3} g^T \Sigma g\\
    \end{aligned}
\end{equation}

Recalling the assumption that $w_m^T g = 0$, we finally have:
\begin{equation}
    \begin{aligned}
        E(\tilde{r}_{1,i}) = \mu_m  - \frac{\delta_i \bar{d}}{a^3} g^T \Sigma g\\
    \end{aligned}
\end{equation}


\paragraph{Proposition 2.} \textit{The mean of the excess return on 
investor $i$'s portfolio is given by:}

\begin{equation}
    E(\tilde{r}_{1,i}) = \mu_m - \frac{\delta_i \bar{d}}{a^3} g^T \Sigma g
\end{equation}

\textit{Investor $i$ with $\delta_i > 0$ accepts below-market expected returns 
in exchange for satisfying their stronger tastes for holding green stocks.
Conversely, and as a result, investor $i$ with $\delta_i < 0$ enjoys above-market expected returns.}


The variance of the excess return on investor $i$'s portfolio is:

\begin{equation}
    \text{Var}(\tilde{r}_{1,i}) = X_i^T \Sigma X_i
\end{equation}

Again, we can express $X_i$ in terms of $w_m$ by susbtracting the expression $w_m$
from the expression of $X_i$, then distribute:

\begin{equation}
    \begin{aligned}
        \text{Var}(\tilde{r}_{1,i}) = (w^T_m + \frac{\delta_i}{a^2} \Sigma^{-1}g) \Sigma (w^T_m + \frac{\delta_i}{a^2} \Sigma^{-1}g) \\
        = w^T_m \Sigma w_m + w^T_m \Sigma \frac{\delta_i}{a^2} \Sigma^{-1}g + w^T_m \Sigma \frac{\delta_i}{a^2} \Sigma^{-1}g + \frac{\delta_i^2}{a^4} g^T \Sigma^{-1} \Sigma \Sigma^{-1} g \\
        = w^T_m \Sigma w_m + w^T_m \frac{\delta_i}{a^2} g + w^T_m \frac{\delta_i}{a^2} g + \frac{\delta_i^2}{a^4} g^T \Sigma^{-1} g \\
    \end{aligned}
\end{equation}

Finally, we recall that $w^T_m \Sigma w_m = \sigma^2_m$ and the assumption 
that $w_m^T g = 0$, then we have:

\begin{equation}
    \begin{aligned}
        \text{Var}(\tilde{r}_{1,i}) = \sigma^2_m + \frac{\delta_i^2}{a^4} g^T \Sigma^{-1} g \\
    \end{aligned}
\end{equation}

\paragraph{Proposition 3.} \textit{The variance of the excess return on
investor $i$'s portfolio is given by:}

\begin{equation}
    \text{Var}(\tilde{r}_{1,i}) = \sigma^2_m + \frac{\delta_i^2}{a^4} g^T \Sigma^{-1} g
\end{equation}

\textit{In departing from the market portfolio, all agents with $\delta_i \neq 0$
incur higher volatility than that of the market portfolio.}

\subsection{Investor's Utility in Equilibrium}

The lower expected returns earned by ESG-oriented investors 
do not imply that these agents are unhappy. Indeed, 
the more an investor's ESG preferences $d_i$ differ from the 
average in either direction, the more 
ESG preferences contribute to the investor's utility.
To see this, we start again from the investor's expected utility:


\begin{equation}
    \begin{aligned}
    E_0(V(\tilde{W}_{1,i}, X_i)) = 
     -\exp{(-a_i (1 + r_f))} \exp{(-a_i X_i^T (\mu + \frac{b_i}{a_i})+\frac{1}{2}a_i^2 X_i^T \Sigma X_i)} \\
    \end{aligned}
\end{equation}

In the second exponent term, we know from the equation 
of the investor's expected excess returns that (and recalling the assumption $a_i = a$):

\begin{equation}
    \begin{aligned}
    -a_iX_i^T \mu = -a (\mu_m - \frac{\delta_i \bar{d}}{a^3} g^T \Sigma g) \\
    = -a \mu_m + \frac{\delta_i \bar{d}}{a^2} g^T \Sigma g
    \end{aligned}
\end{equation}

We have the term $-ai X_i^T \frac{b_i}{a_i} = -X_i^T b_i$, where we 
again can express $X_i$ in terms of $w_m$ and recall that $b_i = d_i g$ and 
the assumption that $w_m^T g = 0$:

\begin{equation}
    \begin{aligned}
    -X_i^T b_i = -X_i^T d_i g \\
    = - (w_m^T + \frac{\delta_i}{a^2} g^T \Sigma^{-1}) d_i g \\
    = -w_m^T d_i g - \frac{\delta_i}{a^2} g^T \Sigma^{-1} d_i g \\
    = - \frac{\delta_i}{a^2} g^T \Sigma^{-1} d_i g
    \end{aligned}
\end{equation}

And we have finally the term $\frac{1}{2}a_i^2 X_i^T \Sigma X_i$,
where we recognize $X_i^T \Sigma X_i$ that we have found earlier:

\begin{equation}
    \begin{aligned}
    \frac{1}{2}a_i^2 X_i^T \Sigma X_i = \frac{1}{2}a_i^2 (w_m^T + \frac{\delta_i}{a^2} g^T \Sigma^{-1}) \Sigma (w_m + \frac{\delta_i}{a^2} g^T \Sigma^{-1}) \\
        = \frac{a^2}{2}(\sigma^2_m + \frac{\delta_i^2}{a^4} g^T \Sigma^{-1} g) \\
        = \frac{a^2}{2} \sigma^2_m + \frac{\delta_i^2}{2a^2} g^T \Sigma^{-1} g
    \end{aligned}
\end{equation}

Adding the three terms together, we have:

\begin{equation}
    \begin{aligned}
        -a_i X^T \mu - X_i^T b_i + (\frac{a^2}{2})X_i^T \Sigma X_i \\
        = -a \mu_m + \frac{\delta_i \bar{d}}{a^2} g^T \Sigma g -  \frac{\delta_i}{a^2} g^T \Sigma^{-1} d_i g + \frac{a^2}{2} \sigma^2_m + \frac{\delta_i^2}{2a^2} g^T \Sigma^{-1} g \\
    \end{aligned}
\end{equation}

\section{ESG Portfolio}

\subsection{Portfolio Tilts}


\subsection{Factor Pricing with the ESG Portfolio}