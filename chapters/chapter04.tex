\chapter{Performance Measures}

We can use mainly two measures to evaluate the performance of the mimicking portfolio. 

\section{Some Empirical Methods}

EMPIRICAL NOTES

\section{Factors Exposure and Risk Premia}

\section{Climate Risk Exposure and Risk Premium}

The first one is the \textbf{$\beta$ (exposure) to the innovation $\Delta E_t(CC_{t+k})$}.
We can regress the mimicking portfolio returns on the 
climate news and the factors returns:

\begin{equation}
    R_{CEP,t} = \beta \Delta E_t(CC_{t+k}) + \gamma^T F_t + \epsilon_t
\end{equation}

We expect the estimated $\beta$ to be positive and the bigger the better.
This regression should be run as an \textit{out-of-sample} test (i.e. the regression is run on a period that is not used to estimate the signal $b$).


The second measure is \textbf{the $\alpha$ (risk premium) of the mimicking portfolio}:

\begin{equation}
    R_{CEP,t} - rf_t = \alpha + \gamma^T F_t + \epsilon_t
\end{equation}

The estimated $\alpha$ should be positive and the bigger the better.

\section{Further Reading}

\section{Exercises}

\section{Solutions}

\section{Python Project}